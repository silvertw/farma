% Generated by Sphinx.
\def\sphinxdocclass{report}
\newif\ifsphinxKeepOldNames \sphinxKeepOldNamestrue
\documentclass[a4paper,10pt,spanish]{sphinxmanual}
\usepackage{iftex}

\ifPDFTeX
  \usepackage[utf8]{inputenc}
\fi
\ifdefined\DeclareUnicodeCharacter
  \DeclareUnicodeCharacter{00A0}{\nobreakspace}
\fi
\usepackage{cmap}
\usepackage[T1]{fontenc}
\usepackage{amsmath,amssymb,amstext}
\usepackage{babel}
\usepackage{times}
\usepackage[Sonny]{fncychap}
\usepackage{longtable}
\usepackage{sphinx}
\usepackage{multirow}
\usepackage{eqparbox}


\addto\captionsspanish{\renewcommand{\figurename}{Figura }}
\addto\captionsspanish{\renewcommand{\tablename}{Tabla }}
\SetupFloatingEnvironment{literal-block}{name=Lista }

\addto\extrasspanish{\def\pageautorefname{página}}

\setcounter{tocdepth}{2}


\title{Farma Plus - Manual de Usuario}
\date{15 de febrero de 2017}
\release{}
\author{Conti, Martin - Williams, Adrian}
\newcommand{\sphinxlogo}{\sphinxincludegraphics{logopequeno2.png}\par}
\renewcommand{\releasename}{Publicación}
\makeindex

\makeatletter
\def\PYG@reset{\let\PYG@it=\relax \let\PYG@bf=\relax%
    \let\PYG@ul=\relax \let\PYG@tc=\relax%
    \let\PYG@bc=\relax \let\PYG@ff=\relax}
\def\PYG@tok#1{\csname PYG@tok@#1\endcsname}
\def\PYG@toks#1+{\ifx\relax#1\empty\else%
    \PYG@tok{#1}\expandafter\PYG@toks\fi}
\def\PYG@do#1{\PYG@bc{\PYG@tc{\PYG@ul{%
    \PYG@it{\PYG@bf{\PYG@ff{#1}}}}}}}
\def\PYG#1#2{\PYG@reset\PYG@toks#1+\relax+\PYG@do{#2}}

\expandafter\def\csname PYG@tok@gd\endcsname{\def\PYG@tc##1{\textcolor[rgb]{0.63,0.00,0.00}{##1}}}
\expandafter\def\csname PYG@tok@gu\endcsname{\let\PYG@bf=\textbf\def\PYG@tc##1{\textcolor[rgb]{0.50,0.00,0.50}{##1}}}
\expandafter\def\csname PYG@tok@gt\endcsname{\def\PYG@tc##1{\textcolor[rgb]{0.00,0.27,0.87}{##1}}}
\expandafter\def\csname PYG@tok@gs\endcsname{\let\PYG@bf=\textbf}
\expandafter\def\csname PYG@tok@gr\endcsname{\def\PYG@tc##1{\textcolor[rgb]{1.00,0.00,0.00}{##1}}}
\expandafter\def\csname PYG@tok@cm\endcsname{\let\PYG@it=\textit\def\PYG@tc##1{\textcolor[rgb]{0.25,0.50,0.56}{##1}}}
\expandafter\def\csname PYG@tok@vg\endcsname{\def\PYG@tc##1{\textcolor[rgb]{0.73,0.38,0.84}{##1}}}
\expandafter\def\csname PYG@tok@vi\endcsname{\def\PYG@tc##1{\textcolor[rgb]{0.73,0.38,0.84}{##1}}}
\expandafter\def\csname PYG@tok@mh\endcsname{\def\PYG@tc##1{\textcolor[rgb]{0.13,0.50,0.31}{##1}}}
\expandafter\def\csname PYG@tok@cs\endcsname{\def\PYG@tc##1{\textcolor[rgb]{0.25,0.50,0.56}{##1}}\def\PYG@bc##1{\setlength{\fboxsep}{0pt}\colorbox[rgb]{1.00,0.94,0.94}{\strut ##1}}}
\expandafter\def\csname PYG@tok@ge\endcsname{\let\PYG@it=\textit}
\expandafter\def\csname PYG@tok@vc\endcsname{\def\PYG@tc##1{\textcolor[rgb]{0.73,0.38,0.84}{##1}}}
\expandafter\def\csname PYG@tok@il\endcsname{\def\PYG@tc##1{\textcolor[rgb]{0.13,0.50,0.31}{##1}}}
\expandafter\def\csname PYG@tok@go\endcsname{\def\PYG@tc##1{\textcolor[rgb]{0.20,0.20,0.20}{##1}}}
\expandafter\def\csname PYG@tok@cp\endcsname{\def\PYG@tc##1{\textcolor[rgb]{0.00,0.44,0.13}{##1}}}
\expandafter\def\csname PYG@tok@gi\endcsname{\def\PYG@tc##1{\textcolor[rgb]{0.00,0.63,0.00}{##1}}}
\expandafter\def\csname PYG@tok@gh\endcsname{\let\PYG@bf=\textbf\def\PYG@tc##1{\textcolor[rgb]{0.00,0.00,0.50}{##1}}}
\expandafter\def\csname PYG@tok@ni\endcsname{\let\PYG@bf=\textbf\def\PYG@tc##1{\textcolor[rgb]{0.84,0.33,0.22}{##1}}}
\expandafter\def\csname PYG@tok@nl\endcsname{\let\PYG@bf=\textbf\def\PYG@tc##1{\textcolor[rgb]{0.00,0.13,0.44}{##1}}}
\expandafter\def\csname PYG@tok@nn\endcsname{\let\PYG@bf=\textbf\def\PYG@tc##1{\textcolor[rgb]{0.05,0.52,0.71}{##1}}}
\expandafter\def\csname PYG@tok@no\endcsname{\def\PYG@tc##1{\textcolor[rgb]{0.38,0.68,0.84}{##1}}}
\expandafter\def\csname PYG@tok@na\endcsname{\def\PYG@tc##1{\textcolor[rgb]{0.25,0.44,0.63}{##1}}}
\expandafter\def\csname PYG@tok@nb\endcsname{\def\PYG@tc##1{\textcolor[rgb]{0.00,0.44,0.13}{##1}}}
\expandafter\def\csname PYG@tok@nc\endcsname{\let\PYG@bf=\textbf\def\PYG@tc##1{\textcolor[rgb]{0.05,0.52,0.71}{##1}}}
\expandafter\def\csname PYG@tok@nd\endcsname{\let\PYG@bf=\textbf\def\PYG@tc##1{\textcolor[rgb]{0.33,0.33,0.33}{##1}}}
\expandafter\def\csname PYG@tok@ne\endcsname{\def\PYG@tc##1{\textcolor[rgb]{0.00,0.44,0.13}{##1}}}
\expandafter\def\csname PYG@tok@nf\endcsname{\def\PYG@tc##1{\textcolor[rgb]{0.02,0.16,0.49}{##1}}}
\expandafter\def\csname PYG@tok@si\endcsname{\let\PYG@it=\textit\def\PYG@tc##1{\textcolor[rgb]{0.44,0.63,0.82}{##1}}}
\expandafter\def\csname PYG@tok@s2\endcsname{\def\PYG@tc##1{\textcolor[rgb]{0.25,0.44,0.63}{##1}}}
\expandafter\def\csname PYG@tok@nt\endcsname{\let\PYG@bf=\textbf\def\PYG@tc##1{\textcolor[rgb]{0.02,0.16,0.45}{##1}}}
\expandafter\def\csname PYG@tok@nv\endcsname{\def\PYG@tc##1{\textcolor[rgb]{0.73,0.38,0.84}{##1}}}
\expandafter\def\csname PYG@tok@s1\endcsname{\def\PYG@tc##1{\textcolor[rgb]{0.25,0.44,0.63}{##1}}}
\expandafter\def\csname PYG@tok@ch\endcsname{\let\PYG@it=\textit\def\PYG@tc##1{\textcolor[rgb]{0.25,0.50,0.56}{##1}}}
\expandafter\def\csname PYG@tok@m\endcsname{\def\PYG@tc##1{\textcolor[rgb]{0.13,0.50,0.31}{##1}}}
\expandafter\def\csname PYG@tok@gp\endcsname{\let\PYG@bf=\textbf\def\PYG@tc##1{\textcolor[rgb]{0.78,0.36,0.04}{##1}}}
\expandafter\def\csname PYG@tok@sh\endcsname{\def\PYG@tc##1{\textcolor[rgb]{0.25,0.44,0.63}{##1}}}
\expandafter\def\csname PYG@tok@ow\endcsname{\let\PYG@bf=\textbf\def\PYG@tc##1{\textcolor[rgb]{0.00,0.44,0.13}{##1}}}
\expandafter\def\csname PYG@tok@sx\endcsname{\def\PYG@tc##1{\textcolor[rgb]{0.78,0.36,0.04}{##1}}}
\expandafter\def\csname PYG@tok@bp\endcsname{\def\PYG@tc##1{\textcolor[rgb]{0.00,0.44,0.13}{##1}}}
\expandafter\def\csname PYG@tok@c1\endcsname{\let\PYG@it=\textit\def\PYG@tc##1{\textcolor[rgb]{0.25,0.50,0.56}{##1}}}
\expandafter\def\csname PYG@tok@o\endcsname{\def\PYG@tc##1{\textcolor[rgb]{0.40,0.40,0.40}{##1}}}
\expandafter\def\csname PYG@tok@kc\endcsname{\let\PYG@bf=\textbf\def\PYG@tc##1{\textcolor[rgb]{0.00,0.44,0.13}{##1}}}
\expandafter\def\csname PYG@tok@c\endcsname{\let\PYG@it=\textit\def\PYG@tc##1{\textcolor[rgb]{0.25,0.50,0.56}{##1}}}
\expandafter\def\csname PYG@tok@mf\endcsname{\def\PYG@tc##1{\textcolor[rgb]{0.13,0.50,0.31}{##1}}}
\expandafter\def\csname PYG@tok@err\endcsname{\def\PYG@bc##1{\setlength{\fboxsep}{0pt}\fcolorbox[rgb]{1.00,0.00,0.00}{1,1,1}{\strut ##1}}}
\expandafter\def\csname PYG@tok@mb\endcsname{\def\PYG@tc##1{\textcolor[rgb]{0.13,0.50,0.31}{##1}}}
\expandafter\def\csname PYG@tok@ss\endcsname{\def\PYG@tc##1{\textcolor[rgb]{0.32,0.47,0.09}{##1}}}
\expandafter\def\csname PYG@tok@sr\endcsname{\def\PYG@tc##1{\textcolor[rgb]{0.14,0.33,0.53}{##1}}}
\expandafter\def\csname PYG@tok@mo\endcsname{\def\PYG@tc##1{\textcolor[rgb]{0.13,0.50,0.31}{##1}}}
\expandafter\def\csname PYG@tok@kd\endcsname{\let\PYG@bf=\textbf\def\PYG@tc##1{\textcolor[rgb]{0.00,0.44,0.13}{##1}}}
\expandafter\def\csname PYG@tok@mi\endcsname{\def\PYG@tc##1{\textcolor[rgb]{0.13,0.50,0.31}{##1}}}
\expandafter\def\csname PYG@tok@kn\endcsname{\let\PYG@bf=\textbf\def\PYG@tc##1{\textcolor[rgb]{0.00,0.44,0.13}{##1}}}
\expandafter\def\csname PYG@tok@cpf\endcsname{\let\PYG@it=\textit\def\PYG@tc##1{\textcolor[rgb]{0.25,0.50,0.56}{##1}}}
\expandafter\def\csname PYG@tok@kr\endcsname{\let\PYG@bf=\textbf\def\PYG@tc##1{\textcolor[rgb]{0.00,0.44,0.13}{##1}}}
\expandafter\def\csname PYG@tok@s\endcsname{\def\PYG@tc##1{\textcolor[rgb]{0.25,0.44,0.63}{##1}}}
\expandafter\def\csname PYG@tok@kp\endcsname{\def\PYG@tc##1{\textcolor[rgb]{0.00,0.44,0.13}{##1}}}
\expandafter\def\csname PYG@tok@w\endcsname{\def\PYG@tc##1{\textcolor[rgb]{0.73,0.73,0.73}{##1}}}
\expandafter\def\csname PYG@tok@kt\endcsname{\def\PYG@tc##1{\textcolor[rgb]{0.56,0.13,0.00}{##1}}}
\expandafter\def\csname PYG@tok@sc\endcsname{\def\PYG@tc##1{\textcolor[rgb]{0.25,0.44,0.63}{##1}}}
\expandafter\def\csname PYG@tok@sb\endcsname{\def\PYG@tc##1{\textcolor[rgb]{0.25,0.44,0.63}{##1}}}
\expandafter\def\csname PYG@tok@k\endcsname{\let\PYG@bf=\textbf\def\PYG@tc##1{\textcolor[rgb]{0.00,0.44,0.13}{##1}}}
\expandafter\def\csname PYG@tok@se\endcsname{\let\PYG@bf=\textbf\def\PYG@tc##1{\textcolor[rgb]{0.25,0.44,0.63}{##1}}}
\expandafter\def\csname PYG@tok@sd\endcsname{\let\PYG@it=\textit\def\PYG@tc##1{\textcolor[rgb]{0.25,0.44,0.63}{##1}}}

\def\PYGZbs{\char`\\}
\def\PYGZus{\char`\_}
\def\PYGZob{\char`\{}
\def\PYGZcb{\char`\}}
\def\PYGZca{\char`\^}
\def\PYGZam{\char`\&}
\def\PYGZlt{\char`\<}
\def\PYGZgt{\char`\>}
\def\PYGZsh{\char`\#}
\def\PYGZpc{\char`\%}
\def\PYGZdl{\char`\$}
\def\PYGZhy{\char`\-}
\def\PYGZsq{\char`\'}
\def\PYGZdq{\char`\"}
\def\PYGZti{\char`\~}
% for compatibility with earlier versions
\def\PYGZat{@}
\def\PYGZlb{[}
\def\PYGZrb{]}
\makeatother

\renewcommand\PYGZsq{\textquotesingle}

\begin{document}
\shorthandoff{"}
\maketitle
\tableofcontents
\phantomsection\label{index::doc}



\chapter{Pantalla Principal}
\label{pantallaprincipal::doc}\label{pantallaprincipal:farma-plus}\label{pantallaprincipal:pantalla-principal}
La pantalla principal del sistema ofrece un menú de navegación en el que se encuentran las gestiones generales del sistema.

\noindent{\hspace*{\fill}\sphinxincludegraphics{{pantallaprincipal}.png}\hspace*{\fill}}

Estas gestiones son:
\begin{itemize}
\item {} 
{\hyperref[pantallaprincipal:gestion\string-usuarios]{\sphinxcrossref{\DUrole{std,std-ref}{Gestíón de Usuarios}}}}

\item {} 
{\hyperref[pantallaprincipal:gestion\string-medicamentos]{\sphinxcrossref{\DUrole{std,std-ref}{Gestión de Medicamentos}}}}

\item {} 
{\hyperref[pantallaprincipal:gestion\string-organizaciones]{\sphinxcrossref{\DUrole{std,std-ref}{Gestión de Organizaciones}}}}

\item {} 
{\hyperref[pantallaprincipal:gestion\string-pedidos]{\sphinxcrossref{\DUrole{std,std-ref}{Gestión de Pedidos}}}}

\item {} 
{\hyperref[pantallaprincipal:gestion\string-facturacion]{\sphinxcrossref{\DUrole{std,std-ref}{Gestión de Facturacion}}}}

\item {} 
{\hyperref[pantallaprincipal:gestion\string-stockdistribuido]{\sphinxcrossref{\DUrole{std,std-ref}{Gestión de Stock Distribuido}}}}

\item {} 
{\hyperref[pantallaprincipal:gestion\string-informeventas]{\sphinxcrossref{\DUrole{std,std-ref}{Gestión de Informe de Ventas}}}}

\end{itemize}


\section{Gestión de Usuarios}
\label{pantallaprincipal:gestion-de-usuarios}\label{pantallaprincipal:gestion-usuarios}
La \emph{Gestión de Usuarios} muestra la información relacionada al usuario activo y funcionalidades que esten acordes a los permisos que posea el mismo.

\noindent{\hspace*{\fill}\sphinxincludegraphics{{gestionusuarios}.png}\hspace*{\fill}}
\begin{description}
\item[{Estas funcionalidades son:}] \leavevmode\begin{itemize}
\item {} 
Entrar al Sistema

\item {} 
Agregar Usuario

\item {} 
Cerrar Sesión

\end{itemize}

\end{description}


\section{Gestión de Medicamentos}
\label{pantallaprincipal:gestion-medicamentos}\label{pantallaprincipal:gestion-de-medicamentos}
La \emph{Gestión de Medicamentos} muestra las funcionalidades relacionadas a los medicamentos que manejará el sistema.

\noindent{\hspace*{\fill}\sphinxincludegraphics{{gestionmedicamentos}.png}\hspace*{\fill}}
\begin{description}
\item[{Estas funcionalidades estan divididas en estas cuatro secciones:}] \leavevmode\begin{itemize}
\item {} 
Medicamentos

\item {} 
Monodrogas

\item {} 
Nombres Fantasía

\item {} 
Presentaciones

\end{itemize}

\end{description}


\section{Gestión de Organizaciones}
\label{pantallaprincipal:gestion-de-organizaciones}\label{pantallaprincipal:gestion-organizaciones}
La \emph{Gestión de Organizaciones} muestra las funcionalidades relacionadas a las organizaciones que manejará el sistema.

\noindent{\hspace*{\fill}\sphinxincludegraphics{{gestionorganizaciones}.png}\hspace*{\fill}}
\begin{description}
\item[{Estas funcionalidades estan divididas en estas cuatro secciones:}] \leavevmode\begin{itemize}
\item {} 
Farmacias

\item {} 
Clínicas

\item {} 
Laboratorios

\item {} 
Obras Sociales

\end{itemize}

\end{description}


\section{Gestión de Pedidos}
\label{pantallaprincipal:gestion-de-pedidos}\label{pantallaprincipal:gestion-pedidos}
La \emph{Gestión de Pedidos} muestra las funcionalidades relacionadas a los pedidos que manejará el sistema.

\noindent{\hspace*{\fill}\sphinxincludegraphics{{gestionpedidos}.png}\hspace*{\fill}}
\begin{description}
\item[{Estas funcionalidades estan divididas en estas cinco secciones:}] \leavevmode\begin{itemize}
\item {} 
Pedido de Farmacia

\item {} 
Pedido de Clínica

\item {} 
Pedido a Laboratorio

\item {} 
Recepción Pedido a Laboratorio

\item {} 
Devolución de Medicamentos Vencidos

\end{itemize}

\end{description}


\section{Gestión de Facturacion}
\label{pantallaprincipal:gestion-de-facturacion}\label{pantallaprincipal:gestion-facturacion}
La \emph{Facturacion} muestra las funcionalidades relacionadas a la Facturacion que manejará el sistema.

\noindent{\hspace*{\fill}\sphinxincludegraphics{{gestionfacturacion}.png}\hspace*{\fill}}
\begin{description}
\item[{Estas funcionalidades estan divididas en estas una seccion:}] \leavevmode\begin{itemize}
\item {} 
Ventas

\item {} 
Compras

\item {} 
Formas de Pago

\end{itemize}

\end{description}


\section{Gestión de Stock Distribuido}
\label{pantallaprincipal:gestion-de-stock-distribuido}\label{pantallaprincipal:gestion-stockdistribuido}
El \emph{Stock Distribuido} muestra las funcionalidades relacionadas a el Stock que manejará el sistema.

\noindent{\hspace*{\fill}\sphinxincludegraphics{{gestionstock}.png}\hspace*{\fill}}
\begin{description}
\item[{Estas funcionalidades estan divididas en estas una seccion:}] \leavevmode\begin{itemize}
\item {} 
Ver

\end{itemize}

\end{description}


\section{Gestión de Informe de Ventas}
\label{pantallaprincipal:gestion-de-informe-de-ventas}\label{pantallaprincipal:gestion-informeventas}
El \emph{Informe de Ventas} muestra las funcionalidades relacionadas a los informes de ventas que manejará el sistema.

\noindent{\hspace*{\fill}\sphinxincludegraphics{{gestionventas}.png}\hspace*{\fill}}
\begin{description}
\item[{Estas funcionalidades estan divididas en estas una seccion:}] \leavevmode\begin{itemize}
\item {} 
Ver

\end{itemize}

\end{description}


\chapter{Usuarios}
\label{usuarios:usuarios}\label{usuarios::doc}
Luego de hacer click en el ítem con el nombre de usuario activo del menú principal el sistema muestra un submenú donde el usuario puede seleccionar la actividad que desea realizar:


\section{Entrar al sistema}
\label{entraralsistema::doc}\label{entraralsistema:entrar-al-sistema}
Se presentará una pantalla que contendrá un formulario de logueo en el que el usuario deberá ingresar los datos de su cuenta.

\noindent{\hspace*{\fill}\sphinxincludegraphics{{pantallalogin}.png}\hspace*{\fill}}

\begin{notice}{attention}{Atención:}
El sistema siempre validará que la información ingresada sea correcta. En caso de que los datos ingresados sean incorrectos el sistema lo informará.
En este punto, las posibles causas de errores son:
\begin{itemize}
\item {} 
Uno o más campos obligatorios vacíos.

\item {} 
El usuario ingresado no existe en el sistema.

\item {} 
La contraseña ingresada es incorrecta.

\end{itemize}
\end{notice}

Una vez completado el formulario, el usuario tendrá que presionar el botón \sphinxcode{Acceder} y será redirigido a la pantalla principal del sistema.

\noindent{\hspace*{\fill}\sphinxincludegraphics{{pantallaprincipal}.png}\hspace*{\fill}}


\section{Agregar Usuario}
\label{agregarusuario::doc}\label{agregarusuario:agregar-usuario}
Si el usuario desea crear un nuevo \emph{Usuario}, deberá presionar el sub-item \sphinxcode{Agregar Usuario}.

\noindent{\hspace*{\fill}\sphinxincludegraphics{{menuagregarusuario}.png}\hspace*{\fill}}

A continuación el sistema lo redirigirá a la siguiente pantalla:

\noindent{\hspace*{\fill}\sphinxincludegraphics{{addusuario}.png}\hspace*{\fill}}

En esta parte el usuario se le presentará un formulario y deberá ingresar los datos solicitados para dar de alta un nuevo \emph{Usuario}.

\begin{notice}{attention}{Atención:}
El sistema siempre validará que la información ingresada sea correcta. En caso de que los datos ingresados sean incorrectos el sistema lo informará.
En este punto, las posibles causas de errores son:
\begin{itemize}
\item {} 
Uno o más campos vacíos.

\item {} 
Uno o más campos con un formato incorrecto.

\item {} 
El nombre de usuario ya existe.

\item {} 
Las contraseñas no coinciden.

\end{itemize}
\end{notice}

Una vez completado el formulario, el usuario tendrá que presionar el boton \sphinxcode{Registrar} y el sistema se encargará de dar de alta el nuevo usuario.


\section{Cerrar Sesión}
\label{cerrarsesion:cerrar-sesion}\label{cerrarsesion::doc}
Si el usuario desea desloguearse del sistema, deberá presionar el sub-item \sphinxcode{Cerrar Sesión}.

\noindent{\hspace*{\fill}\sphinxincludegraphics{{cerrarsesion}.png}\hspace*{\fill}}

El sistema se encargará de \textbf{Cerrar la Sesión} del usuario y redirigirlo a la pantalla de ingreso del sistema.......

\noindent{\hspace*{\fill}\sphinxincludegraphics{{pantallalogin}.png}\hspace*{\fill}}


\chapter{Medicamentos}
\label{medicamentos::doc}\label{medicamentos:medicamentos}
Luego de hacer ``click'' en el ítem “Medicamentos” del menú principal el sistema muestra un submenú donde el usuario puede seleccionar la actividad que desea realizar:


\section{Medicamentos}
\label{medicams::doc}\label{medicams:medicamentos}
Se presentará una pantalla que contendrá un listado con todos los \emph{Medicamentos} que se encuentren registrados en el sistema hasta la fecha.

\noindent{\hspace*{\fill}\sphinxincludegraphics{{medicamentos}.png}\hspace*{\fill}}

Junto con el listado, se ofrecerán un conjunto de funcionalidades que permitirán manipular estos \emph{Medicamentos}

Estas funcionalidades son:
\begin{itemize}
\item {} 
{\hyperref[medicams:alta\string-medicamento]{\sphinxcrossref{\DUrole{std,std-ref}{Alta Medicamento}}}}

\item {} 
{\hyperref[medicams:modificar\string-stock\string-minimo]{\sphinxcrossref{\DUrole{std,std-ref}{Modificar Stock Mínimo}}}}

\item {} 
{\hyperref[medicams:modificar\string-precio\string-venta]{\sphinxcrossref{\DUrole{std,std-ref}{Modificar Precio Venta}}}}

\item {} 
{\hyperref[medicams:eliminar\string-medicamento]{\sphinxcrossref{\DUrole{std,std-ref}{Eliminar Medicamento}}}}

\item {} 
{\hyperref[medicams:ver\string-lotes]{\sphinxcrossref{\DUrole{std,std-ref}{Ver Lotes}}}}

\item {} 
{\hyperref[medicams:stock\string-global]{\sphinxcrossref{\DUrole{std,std-ref}{Stock Global}}}}

\item {} 
{\hyperref[medicams:formulario\string-busqueda\string-medicamento]{\sphinxcrossref{\DUrole{std,std-ref}{Formulario de Búsqueda}}}}

\item {} \begin{description}
\item[{{\hyperref[pedidosclinica:reportes\string-pc]{\sphinxcrossref{\DUrole{std,std-ref}{Reportes}}}}}] \leavevmode\begin{itemize}
\item {} 
{\hyperref[medicams:meds\string-cant\string-pc]{\sphinxcrossref{\DUrole{std,std-ref}{Medicamentos mas solicitados por cantidad}}}}

\item {} 
{\hyperref[medicams:meds\string-ped\string-pc]{\sphinxcrossref{\DUrole{std,std-ref}{Medicamentos mas solicitados por pedido}}}}

\item {} 
{\hyperref[medicams:org\string-cant\string-pc]{\sphinxcrossref{\DUrole{std,std-ref}{Organizaciones mas demandantes de un medicamento por cantidad}}}}

\item {} 
{\hyperref[medicams:org\string-ped\string-pc]{\sphinxcrossref{\DUrole{std,std-ref}{Organizaciones mas demandantes de un medicamento por pedido}}}}

\end{itemize}

\end{description}

\end{itemize}


\subsection{Alta Medicamento}
\label{medicams:id1}\label{medicams:alta-medicamento}
Si el usuario desea crear un nuevo \emph{Medicamento}, deberá presionar el botón \sphinxcode{Alta}.

\noindent{\hspace*{\fill}\sphinxincludegraphics{{btnaltamed}.png}\hspace*{\fill}}

A continuación el sistema lo redirigirá a la siguiente pantalla:

\noindent{\hspace*{\fill}\sphinxincludegraphics{{altamed}.jpg}\hspace*{\fill}}

En esta parte al usuario se le presentará un formulario y deberá ingresar los datos solicitados para dar de alta un nuevo \emph{Medicamento}.

\begin{notice}{attention}{Atención:}
El sistema siempre validará que la información ingresada sea correcta. En caso de que los datos ingresados sean incorrectos el sistema lo informará.
En este punto, las posibles causas de errores son:
\begin{itemize}
\item {} 
Uno o más campos vacíos.

\item {} 
El código de barras del medicamento ya existe.

\item {} 
La monodroga ingresada no existe.

\end{itemize}
\end{notice}

Una vez completado el formulario, el usuario tendrá dos opciones:
\begin{itemize}
\item {} 
Presionar el botón \sphinxcode{Guardar y Volver}.

\item {} 
Presionar el botón \sphinxcode{Guardar y Continuar}.

\end{itemize}

El botón \sphinxcode{Guardar y Volver} permite guardar el \emph{Medicamento} en el sistema y volver a la pantalla
principal de medicamentos.

El botón \sphinxcode{Guardar y Continuar} permite guardar el \emph{Medicamento} en el sistema y seguir dando de alta nuevos \emph{Medicamentos}.


\subsection{Modificar Stock Mínimo}
\label{medicams:modificar-stock-minimo}\label{medicams:id2}
Si el usuario desea modificar el stock mínimo de un \emph{Medicamento}, deberá seleccionar el botón de \textbf{Acción} asociado al \emph{Medicamento} y presionar la pestaña \sphinxcode{Modificar Stock Mínimo}.

\noindent{\hspace*{\fill}\sphinxincludegraphics{{modifstockmin}.png}\hspace*{\fill}}

Una vez realizado el paso anterior, el sistema lo redirigirá a la siguiente pantalla:

\noindent{\hspace*{\fill}\sphinxincludegraphics{{modifstockmed}.png}\hspace*{\fill}}

En esta parte el usuario se le presentará un formulario y deberá actualizar la información del stock asociado al \emph{Medicamento}.

\begin{notice}{attention}{Atención:}
El sistema siempre validará que la información ingresada sea correcta. En caso de que los datos ingresados sean incorrectos el sistema lo informará.
En este punto, las posibles causas de errores son:
\begin{itemize}
\item {} 
No se ingresó un stock mínimo.

\item {} 
El stock mínimo ingresado no posee un formato correcto.

\item {} 
El stock mínimo ingresado es menor a cero.

\end{itemize}
\end{notice}

Una vez completado el formulario, el usuario deberá presionar el botón \sphinxcode{Guardar Cambios} y el sistema se encargará de actualizar el stock mínimo del \emph{Medicamento} seleccionado.


\subsection{Modificar Precio de Venta}
\label{medicams:modificar-precio-venta}\label{medicams:modificar-precio-de-venta}
Si el usuario desea modificar el precio de venta de un \emph{Medicamento}, deberá seleccionar el botón de \textbf{Acción} asociado al \emph{Medicamento} y presionar la pestaña \sphinxcode{Modificar Precio Venta}.

\noindent{\hspace*{\fill}\sphinxincludegraphics{{modifprecioventa}.png}\hspace*{\fill}}

Una vez realizado el paso anterior, el sistema lo redirigirá a la siguiente pantalla:

\noindent{\hspace*{\fill}\sphinxincludegraphics{{modifpreciomed}.png}\hspace*{\fill}}

En esta parte el usuario se le presentará un formulario y deberá actualizar la información del precio de venta asociado al \emph{Medicamento}.

\begin{notice}{attention}{Atención:}
El sistema siempre validará que la información ingresada sea correcta. En caso de que los datos ingresados sean incorrectos el sistema lo informará.
En este punto, las posibles causas de errores son:
\begin{itemize}
\item {} 
No se ingresó un precio de venta.

\item {} 
El precio de venta ingresado no posee un formato correcto.

\item {} 
El precio de venta ingresado es menor a cero.

\end{itemize}
\end{notice}

Una vez completado el formulario, el usuario deberá presionar el botón \sphinxcode{Guardar Cambios} y el sistema se encargará de actualizar el precio de venta del \emph{Medicamento} seleccionado.


\subsection{Eliminar Medicamento}
\label{medicams:eliminar-medicamento}\label{medicams:id3}
Si el usuario desea eliminar un \emph{Medicamento}, deberá seleccionar el botón de \textbf{Acción} asociado al \emph{Medicamento} y presionar la pestaña \sphinxcode{Eliminar}.

\noindent{\hspace*{\fill}\sphinxincludegraphics{{btneliminarmed}.png}\hspace*{\fill}}

Una vez realizado el paso anterior aparecerá la siguiente ventana emergente (modal):

\noindent{\hspace*{\fill}\sphinxincludegraphics{{eliminarmed}.png}\hspace*{\fill}}

En esta parte el usuario deberá decidir si confirma la eliminación del \emph{Medicamento} o no. Si desea confirmar la eliminación deberá presionar el botón \sphinxcode{Confirmar}, caso contrario, presionará el botón \sphinxcode{Cancelar}.

\begin{notice}{note}{Nota:}
Aquellos \emph{Medicamentos} que cumplan las siguientes condiciones \textbf{NO} podrán ser eliminados:
\begin{itemize}
\item {} 
Esten pendientes parcial o totalmente en un Pedido a Laboratorio.

\item {} 
Esten pendientes parcial o totalmente en un Pedido de Farmacia.

\item {} 
Posean lotes activos.

\end{itemize}

El sistema se encargará de informar al usuario las razones por las cuales el \emph{Medicamento} seleccionado no puede eliminarse. En dicho caso, el sistema mostrara una ventana emergente (modal) como esta:

\noindent{\hspace*{\fill}\sphinxincludegraphics{{fallaeliminarmed}.png}\hspace*{\fill}}
\end{notice}


\subsection{Ver Lotes}
\label{medicams:id4}\label{medicams:ver-lotes}
Si el usuario desea ver los lotes de un \emph{Medicamento}, deberá seleccionar el botón de \textbf{Acción} asociado al \emph{Medicamento} y presionar la pestaña \sphinxcode{Ver Lotes}.

\noindent{\hspace*{\fill}\sphinxincludegraphics{{verlotes}.png}\hspace*{\fill}}

Una vez realizado el paso anterior aparecerá la siguiente ventana emergente (modal):

\noindent{\hspace*{\fill}\sphinxincludegraphics{{lotesmed}.png}\hspace*{\fill}}

Esta ventana mostrará todos los lotes que estén asociados al \emph{Medicamento}.

\begin{notice}{note}{Nota:}
En caso de que el \emph{Medicamento} seleccionado no posea lotes activos, el sistema se encargará de mostrar la siguiente ventana emergente (modal):

\noindent{\hspace*{\fill}\sphinxincludegraphics{{nolotes}.png}\hspace*{\fill}}
\end{notice}


\subsection{Stock Global}
\label{medicams:stock-global}\label{medicams:id5}
Si el usuario desea ver el Stock de un \emph{Medicamento}, deberá seleccionar el botón de \textbf{Acción} asociado al \emph{Medicamento} y presionar la pestaña \sphinxcode{Stock Global}, el cual es el Stock de la Drogueria mas el de las sucursales.

\noindent{\hspace*{\fill}\sphinxincludegraphics{{btnStockGlobal}.png}\hspace*{\fill}}

Una vez realizado el paso anterior aparecerá la siguiente ventana emergente (modal):

\noindent{\hspace*{\fill}\sphinxincludegraphics{{modalStockGlobal}.png}\hspace*{\fill}}


\subsection{Reportes}
\label{medicams:reportes}\label{medicams:reportes-pc}
Si el usuario desea visualizar y/o generar reportes de estadisticas en relacion a los \emph{Medicamentos}, debera seleccionar el boton de \textbf{Reportes}.

\noindent{\hspace*{\fill}\sphinxincludegraphics{{reportesMedicamentos}.png}\hspace*{\fill}}

Esta funcionalidad cuenta con 4 modalidades:
\begin{itemize}
\item {} 
{\hyperref[medicams:meds\string-cant\string-pc]{\sphinxcrossref{\DUrole{std,std-ref}{Medicamentos mas solicitados por cantidad}}}}

\item {} 
{\hyperref[medicams:meds\string-ped\string-pc]{\sphinxcrossref{\DUrole{std,std-ref}{Medicamentos mas solicitados por pedido}}}}

\item {} 
{\hyperref[medicams:org\string-cant\string-pc]{\sphinxcrossref{\DUrole{std,std-ref}{Organizaciones mas demandantes de un medicamento por cantidad}}}}

\item {} 
{\hyperref[medicams:org\string-ped\string-pc]{\sphinxcrossref{\DUrole{std,std-ref}{Organizaciones mas demandantes de un medicamento por pedido}}}}

\end{itemize}


\subsubsection{Medicamentos mas solicitados por cantidad}
\label{medicams:meds-cant-pc}\label{medicams:medicamentos-mas-solicitados-por-cantidad}
Si el usuario desea que los reportes se generen en base al volumen de medicamentos pedidos por cantidad, debera presionar la opcion \sphinxcode{Medicamentos mas solicitados por cantidad}.

\noindent{\hspace*{\fill}\sphinxincludegraphics{{reportesMedicamentos1}.png}\hspace*{\fill}}

Al hacerlo, se mostrara la siguiente pantalla:

\noindent{\hspace*{\fill}\sphinxincludegraphics{{pantallatop10medsCant1}.png}\hspace*{\fill}}

\noindent{\hspace*{\fill}\sphinxincludegraphics{{pantallatop10medsCant2}.png}\hspace*{\fill}}

Si el usuario desea ajustar el rango de fecha sobre el cual se genera el reporte puede hacerlo utilizando la modalidad de filtrado por fechas:
El usuario tendrá que ingresar los parámetros de búsqueda en el formulario, y presionar el botón \sphinxcode{Filtrar}.

\begin{notice}{note}{Nota:}
Todos los campos son opcionales, de no especificarse ningún criterio de búsqueda el sistema mostrará la informacion historica completa.
\end{notice}

\noindent{\hspace*{\fill}\sphinxincludegraphics{{fechastop10meds1}.png}\hspace*{\fill}}

Si el usuario desea exportar el resultado generado a una planilla de Excel, debera presionar el ícono de excel.

\noindent{\hspace*{\fill}\sphinxincludegraphics{{xlstop10meds2}.png}\hspace*{\fill}}

Si el usuario desea exportar el resultado en un formato de imagen PNG, JPEG, PDF o SVG, debera presionar el boton de herramientas de exportacion y seleccionar la opcion correspondiente.

\noindent{\hspace*{\fill}\sphinxincludegraphics{{btnexptop10meds3}.png}\hspace*{\fill}}


\subsubsection{Medicamentos mas solicitados por pedido}
\label{medicams:medicamentos-mas-solicitados-por-pedido}\label{medicams:meds-ped-pc}
Si el usuario desea que los reportes se generen en base a la cantidad de Medicamentos mas solicitados por pedidos realizados, debera presionar la opcion \sphinxcode{Medicamentos mas solicitados por pedido}.

\noindent{\hspace*{\fill}\sphinxincludegraphics{{reportesMedicamentos2}.png}\hspace*{\fill}}

Al hacerlo, se mostrara la siguiente pantalla:

\noindent{\hspace*{\fill}\sphinxincludegraphics{{pantallatop10medsPed1}.png}\hspace*{\fill}}

\noindent{\hspace*{\fill}\sphinxincludegraphics{{pantallatop10medsPed2}.png}\hspace*{\fill}}

Si el usuario desea ajustar el rango de fecha sobre el cual se genera el reporte puede hacerlo utilizando la modalidad de filtrado por fechas:
El usuario tendrá que ingresar los parámetros de búsqueda en el formulario, y presionar el botón \sphinxcode{Filtrar}.

\begin{notice}{note}{Nota:}
Todos los campos son opcionales, de no especificarse ningún criterio de búsqueda el sistema mostrará la informacion historica completa.
\end{notice}

\noindent{\hspace*{\fill}\sphinxincludegraphics{{fechastop10meds1}.png}\hspace*{\fill}}

Si el usuario desea exportar el resultado generado a una planilla de Excel, debera presionar el ícono de excel.

\noindent{\hspace*{\fill}\sphinxincludegraphics{{xlstop10meds2}.png}\hspace*{\fill}}

Si el usuario desea exportar el resultado en un formato de imagen PNG, JPEG, PDF o SVG, debera presionar el boton de herramientas de exportacion y seleccionar la opcion correspondiente.

\noindent{\hspace*{\fill}\sphinxincludegraphics{{btnexptop10meds3}.png}\hspace*{\fill}}


\subsubsection{Organizaciones mas demandantes de un medicamento por cantidad}
\label{medicams:org-cant-pc}\label{medicams:organizaciones-mas-demandantes-de-un-medicamento-por-cantidad}
Si el usuario desea que los reportes se generen en base a la cantidad de Organizaciones mas demandantes de un medicamento, debera presionar la opcion \sphinxcode{Organizaciones mas demandantes de un medicamento por cantidad}.

\noindent{\hspace*{\fill}\sphinxincludegraphics{{reportesMedicamentos3}.png}\hspace*{\fill}}

Al hacerlo, se mostrara la siguiente pantalla:

\noindent{\hspace*{\fill}\sphinxincludegraphics{{pantallaOrgMedCant}.png}\hspace*{\fill}}

Si el usuario desea ajustar el rango de fecha sobre el cual se genera el reporte puede hacerlo utilizando la modalidad de filtrado por fechas:
El usuario tendrá que ingresar los parámetros de búsqueda en el formulario, y presionar el botón \sphinxcode{Filtrar}.

\noindent{\hspace*{\fill}\sphinxincludegraphics{{pantallaOrgMedCant2}.png}\hspace*{\fill}}

\noindent{\hspace*{\fill}\sphinxincludegraphics{{pantallaOrgMedCant3}.png}\hspace*{\fill}}

\noindent{\hspace*{\fill}\sphinxincludegraphics{{pantallaOrgMedCant4}.png}\hspace*{\fill}}

\noindent{\hspace*{\fill}\sphinxincludegraphics{{pantallaOrgMedCant5}.png}\hspace*{\fill}}

\begin{notice}{note}{Nota:}
Todos los campos son opcionales, de no especificarse ningún criterio de búsqueda el sistema mostrará la informacion historica completa.
\end{notice}

\noindent{\hspace*{\fill}\sphinxincludegraphics{{fechastop10meds1}.png}\hspace*{\fill}}

Si el usuario desea exportar el resultado generado a una planilla de Excel, debera presionar el ícono de excel.

\noindent{\hspace*{\fill}\sphinxincludegraphics{{xlstop10meds2}.png}\hspace*{\fill}}

Si el usuario desea exportar el resultado en un formato de imagen PNG, JPEG, PDF o SVG, debera presionar el boton de herramientas de exportacion y seleccionar la opcion correspondiente.

\noindent{\hspace*{\fill}\sphinxincludegraphics{{btnexptop10meds3}.png}\hspace*{\fill}}


\subsubsection{Organizaciones mas demandantes de un medicamento por pedido}
\label{medicams:organizaciones-mas-demandantes-de-un-medicamento-por-pedido}\label{medicams:org-ped-pc}
Si el usuario desea que los reportes se generen en base a la cantidad de pedidos segun Organizaciones mas demandantes de un medicamento por pedido, debera presionar la opcion \sphinxcode{Organizaciones mas demandantes de un medicamento por pedido}.

\noindent{\hspace*{\fill}\sphinxincludegraphics{{reportesMedicamentos4}.png}\hspace*{\fill}}

Al hacerlo, se mostrara la siguiente pantalla:

\noindent{\hspace*{\fill}\sphinxincludegraphics{{pantallaOrgMedPed}.png}\hspace*{\fill}}

Si el usuario desea ajustar el rango de fecha sobre el cual se genera el reporte puede hacerlo utilizando la modalidad de filtrado por fechas:
El usuario tendrá que ingresar los parámetros de búsqueda en el formulario, y presionar el botón \sphinxcode{Filtrar}.

\noindent{\hspace*{\fill}\sphinxincludegraphics{{pantallaOrgMedCant2}.png}\hspace*{\fill}}

\noindent{\hspace*{\fill}\sphinxincludegraphics{{pantallaOrgMedCant3}.png}\hspace*{\fill}}

\noindent{\hspace*{\fill}\sphinxincludegraphics{{pantallaOrgMedCant4}.png}\hspace*{\fill}}

\noindent{\hspace*{\fill}\sphinxincludegraphics{{pantallaOrgMedCant5}.png}\hspace*{\fill}}

\begin{notice}{note}{Nota:}
Todos los campos son opcionales, de no especificarse ningún criterio de búsqueda el sistema mostrará la informacion historica completa.
\end{notice}

\noindent{\hspace*{\fill}\sphinxincludegraphics{{fechastop10meds1}.png}\hspace*{\fill}}

Si el usuario desea exportar el resultado generado a una planilla de Excel, debera presionar el ícono de excel.

\noindent{\hspace*{\fill}\sphinxincludegraphics{{xlstop10pedsclin}.png}\hspace*{\fill}}

Si el usuario desea exportar el resultado en un formato de imagen PNG, JPEG, PDF o SVG, debera presionar el boton de herramientas de exportacion y seleccionar la opcion correspondiente.

\noindent{\hspace*{\fill}\sphinxincludegraphics{{btnexptop10pedsclin}.png}\hspace*{\fill}}


\subsection{Formulario de Búsqueda}
\label{medicams:formulario-busqueda-medicamento}\label{medicams:formulario-de-busqueda}
Si el usuario desea visualizar sólo aquellos \emph{Medicamentos} que cumplan con algunos criterios en específico, deberá utilizar el formulario de búsqueda.

\noindent{\hspace*{\fill}\sphinxincludegraphics{{busquedamed}.png}\hspace*{\fill}}

Este formulario cuenta con dos modalidades:
\begin{itemize}
\item {} 
Búsqueda simple: permite buscar los \emph{Medicamentos} por nombre fantasía.

\item {} 
Búsqueda avanzada: permite buscar los \emph{Medicamentos} por nombre fantasía y laboratorio.

\end{itemize}

\begin{notice}{note}{Nota:}
Todos los campos son opcionales, de no especificarse ningún criterio de búsqueda el sistema mostrará todos los \emph{Medicamentos}.
\end{notice}

El usuario tendrá que ingresar los parámetros de búsqueda en el formulario, y presionar el botón \sphinxcode{Buscar}. El sistema visualizará aquellos \emph{Medicamentos} que cumplan con todas las condiciones especificadas.

Si el usuario desea limpiar los filtros activos, deberá presionar el boton \sphinxcode{Limpiar}.

\noindent{\hspace*{\fill}\sphinxincludegraphics{{limpiarbusquedamed}.png}\hspace*{\fill}}


\section{Monodrogas}
\label{monodrogas:monodrogas}\label{monodrogas::doc}
Se presentará una pantalla que contendrá un listado con todas las \emph{Monodrogas} que se encuentren registradas en el sistema hasta la fecha.

\noindent{\hspace*{\fill}\sphinxincludegraphics{{monodrogas}.png}\hspace*{\fill}}

Junto con el listado, se ofrecerán un conjunto de funcionalidades que permitirán manipular estas \emph{Monodrogas}

Estas funcionalidades son:
\begin{itemize}
\item {} 
{\hyperref[monodrogas:alta\string-monodroga]{\sphinxcrossref{\DUrole{std,std-ref}{Alta Monodroga}}}}

\item {} 
{\hyperref[monodrogas:modificar\string-monodroga]{\sphinxcrossref{\DUrole{std,std-ref}{Modificar Monodroga}}}}

\item {} 
{\hyperref[monodrogas:eliminar\string-monodroga]{\sphinxcrossref{\DUrole{std,std-ref}{Eliminar Monodroga}}}}

\item {} 
{\hyperref[monodrogas:formulario\string-busqueda\string-monodroga]{\sphinxcrossref{\DUrole{std,std-ref}{Formulario de Búsqueda}}}}

\end{itemize}


\subsection{Alta Monodroga}
\label{monodrogas:alta-monodroga}\label{monodrogas:id1}
Si el usuario desea crear una nueva \emph{Monodroga}, deberá presionar el botón \sphinxcode{Alta}.

\noindent{\hspace*{\fill}\sphinxincludegraphics{{btnaltamono}.png}\hspace*{\fill}}

A continuación el sistema lo redirigirá a la siguiente pantalla:

\noindent{\hspace*{\fill}\sphinxincludegraphics{{altamono}.png}\hspace*{\fill}}

En esta parte el usuario se le presentará un formulario y deberá ingresar los datos solicitados para dar de alta una nueva \emph{Monodroga}.

\begin{notice}{attention}{Atención:}
El sistema siempre validará que la información ingresada sea correcta. En caso de que los datos ingresados sean incorrectos el sistema lo informará.
En este punto, las posibles causas de errores son:
\begin{itemize}
\item {} 
No se ingresó un nombre de monodroga.

\item {} 
El nombre de monodroga contiene caracteres especiales o números.

\item {} 
El nombre de la monodroga ya existe en el sistema.

\end{itemize}
\end{notice}

Una vez completado el formulario, el usuario tendrá dos opciones:
\begin{itemize}
\item {} 
Presionar el botón \sphinxcode{Guardar y Volver}.

\item {} 
Presionar el botón \sphinxcode{Guardar y Continuar}.

\end{itemize}

El botón \sphinxcode{Guardar y Volver} permite guardar la \emph{Monodroga} en el sistema y volver a la pantalla
principal de \emph{Monodrogas}..

El botón \sphinxcode{Guardar y Continuar} permite guardar la \emph{Monodroga} en el sistema y seguir dando de alta nuevas \emph{Monodrogas}.


\subsection{Modificar Monodroga}
\label{monodrogas:modificar-monodroga}\label{monodrogas:id2}
Si el usuario desea modificar los datos de una \emph{Monodroga}, deberá seleccionar el botón de \textbf{Acción} asociado a la \emph{Monodroga} y presionar la pestaña \sphinxcode{Modificar}.

\noindent{\hspace*{\fill}\sphinxincludegraphics{{btnmodificarmono}.png}\hspace*{\fill}}

Una vez realizado el paso anterior, el sistema lo redirigirá a la siguiente pantalla:

\noindent{\hspace*{\fill}\sphinxincludegraphics{{modificarmono}.png}\hspace*{\fill}}

En esta parte al usuario se le presentará un formulario y deberá actualizar los datos asociados a la \emph{Monodroga}.

\begin{notice}{attention}{Atención:}
El sistema siempre validará que la información ingresada sea correcta. En caso de que los datos ingresados sean incorrectos el sistema lo informará.
En este punto, las posibles causas de errores son:
\begin{itemize}
\item {} 
No se ingresó un nombre de monodroga.

\item {} 
El nombre de monodroga contiene caracteres especiales o números.

\item {} 
El nombre de la monodroga ya existe en el sistema.

\end{itemize}
\end{notice}

Una vez completado el formulario, el usuario deberá presionar el botón \sphinxcode{Guardar Cambios} y el sistema se encargara de actualizar los datos de la \emph{Monodroga} seleccionada.


\subsection{Eliminar Monodroga}
\label{monodrogas:id3}\label{monodrogas:eliminar-monodroga}
Si el usuario desea eliminar una \emph{Monodroga}, deberá seleccionar el botón de \textbf{Acción} asociado a la \emph{Monodroga} y presionar la pestaña \sphinxcode{Eliminar}.

\noindent{\hspace*{\fill}\sphinxincludegraphics{{btneliminarmono}.png}\hspace*{\fill}}

Una vez realizado el paso anterior aparecerá la siguiente ventana emergente (modal):

\noindent{\hspace*{\fill}\sphinxincludegraphics{{eliminarmono}.png}\hspace*{\fill}}

En esta parte el usuario deberá decidir si confirma la eliminación de la \emph{Monodroga} o no. Si desea confirmar la eliminación deberá presionar el botón \sphinxcode{Confirmar}, caso contrario, presionará el botón \sphinxcode{Cancelar}.

\begin{notice}{note}{Nota:}
Aquellas \emph{Monodrogas} que cumplan las siguientes condiciones \textbf{NO} podrán ser eliminadas:
\begin{itemize}
\item {} 
Esten asociadas a un medicamento.

\end{itemize}

El sistema se encargará de informar al usuario las razones por las cuales la \emph{Monodroga} seleccionada no puede eliminarse. En dicho caso, el sistema mostrara una ventana emergente (modal) como esta:

\noindent{\hspace*{\fill}\sphinxincludegraphics{{fallaeliminarmono}.png}\hspace*{\fill}}
\end{notice}


\subsection{Formulario de Búsqueda}
\label{monodrogas:formulario-de-busqueda}\label{monodrogas:formulario-busqueda-monodroga}
Si el usuario desea visualizar sólo aquellas \emph{Monodrogas} que cumplan con algunos criterios en específico, deberá utilizar el formulario de búsqueda.

\noindent{\hspace*{\fill}\sphinxincludegraphics{{busquedamono}.png}\hspace*{\fill}}

Este formulario sólo cuenta con la opción de búsqueda simple en base al nombre de la \emph{Monodroga}.

\begin{notice}{note}{Nota:}
Este campo es opcional, de no especificarse ningún criterio de búsqueda el sistema mostrará todas las \emph{Monodrogas}.
\end{notice}

El usuario tendrá que ingresar los parámetros de búsqueda en el formulario, y presionar el botón \sphinxcode{Buscar}. El sistema visualizará aquellas \emph{Monodrogas} que cumplan con todas las condiciones especificadas.


\section{Presentaciones}
\label{presentaciones::doc}\label{presentaciones:presentaciones}
Se presentará una pantalla que contendrá un listado con todas las \emph{Presentaciones} que se encuentren registradas en el sistema hasta la fecha.

\noindent{\hspace*{\fill}\sphinxincludegraphics{{presentaciones}.png}\hspace*{\fill}}

Junto con el listado, se ofrecerán un conjunto de funcionalidades que permitirán manipular estas \emph{Presentaciones}

Estas funcionalidades son:
\begin{itemize}
\item {} 
{\hyperref[presentaciones:alta\string-presentacion]{\sphinxcrossref{\DUrole{std,std-ref}{Alta Presetación}}}}

\item {} 
{\hyperref[presentaciones:modificar\string-presentacion]{\sphinxcrossref{\DUrole{std,std-ref}{Modificar Presetación}}}}

\item {} 
{\hyperref[presentaciones:eliminar\string-presentacion]{\sphinxcrossref{\DUrole{std,std-ref}{Eliminar Presetación}}}}

\item {} 
{\hyperref[presentaciones:formulario\string-busqueda\string-presentacion]{\sphinxcrossref{\DUrole{std,std-ref}{Formulario de Búsqueda}}}}

\end{itemize}


\subsection{Alta Presentación}
\label{presentaciones:alta-presentacion}\label{presentaciones:id1}
Si el usuario desea crear una nueva \emph{Presentación}, deberá presionar el botón \sphinxcode{Alta}.

\noindent{\hspace*{\fill}\sphinxincludegraphics{{btnaltapres}.png}\hspace*{\fill}}

A continuación el sistema lo redirigirá a la siguiente pantalla:

\noindent{\hspace*{\fill}\sphinxincludegraphics{{altapres}.png}\hspace*{\fill}}

En esta parte el usuario se le presentará un formulario y deberá ingresar los datos solicitados para dar de alta una nueva \emph{Presentación}.

\begin{notice}{attention}{Atención:}
El sistema siempre validará que la información ingresada sea correcta. En caso de que los datos ingresados sean incorrectos el sistema lo informará.
En este punto, las posibles causas de errores son:
\begin{itemize}
\item {} 
Uno o más campos vacios.

\item {} 
La cantidad ingresada no posee el formato correcto.

\item {} 
La cantidad ingresada es menor a cero.

\end{itemize}
\end{notice}

Una vez completado el formulario, el usuario tendrá dos opciones:
\begin{itemize}
\item {} 
Presionar el botón \sphinxcode{Guardar y Volver}.

\item {} 
Presionar el botón \sphinxcode{Guardar y Continuar}.

\end{itemize}

El botón \sphinxcode{Guardar y Volver} permite guardar la \emph{Presentación} en el sistema y volver a la pantalla
principal de \emph{Presentaciones}.

El botón \sphinxcode{Guardar y Continuar} permite guardar la \emph{Presentación} en el sistema y seguir dando de alta nuevas \emph{Presentaciones}.


\subsection{Modificar Presentación}
\label{presentaciones:modificar-presentacion}\label{presentaciones:id2}
Si el usuario desea modificar los datos de una \emph{Presentación}, deberá seleccionar el botón de \textbf{Acción} asociado a la \emph{Presentación} y presionar la pestaña \sphinxcode{Modificar}.

\noindent{\hspace*{\fill}\sphinxincludegraphics{{btnmodificarpres}.png}\hspace*{\fill}}

Una vez realizado el paso anterior, el sistema lo redirigirá a la siguiente pantalla:

\noindent{\hspace*{\fill}\sphinxincludegraphics{{modificarpres}.png}\hspace*{\fill}}

En esta parte al usuario se le presentará un formulario y deberá actualizar los datos asociados a la \emph{Presentación}.

\begin{notice}{attention}{Atención:}
El sistema siempre validará que la información ingresada sea correcta. En caso de que los datos ingresados sean incorrectos el sistema lo informará.
En este punto, las posibles causas de errores son:
\begin{itemize}
\item {} 
Uno o más campos vacios.

\item {} 
La cantidad ingresada no posee el formato correcto.

\item {} 
La cantidad ingresada es menor a cero.

\end{itemize}
\end{notice}

Una vez completado el formulario, el usuario deberá presionar el botón \sphinxcode{Guardar Cambios} y el sistema se encargara de actualizar los datos de la \emph{Presentación} seleccionada.


\subsection{Eliminar Presentación}
\label{presentaciones:id3}\label{presentaciones:eliminar-presentacion}
Si el usuario desea eliminar una \emph{Presentación}, deberá seleccionar el botón de \textbf{Acción} asociado a la \emph{Presentación} y presionar la pestaña \sphinxcode{Eliminar}.

\noindent{\hspace*{\fill}\sphinxincludegraphics{{btneliminarpres}.png}\hspace*{\fill}}

Una vez realizado el paso anterior aparecerá la siguiente ventana emergente (modal):

\noindent{\hspace*{\fill}\sphinxincludegraphics{{eliminarpres}.png}\hspace*{\fill}}

En esta parte el usuario deberá decidir si confirma la eliminación de la \emph{Presentación} o no. Si desea confirmar la eliminación deberá presionar el botón \sphinxcode{Confirmar}, caso contrario, presionará el botón \sphinxcode{Cancelar}.

\begin{notice}{note}{Nota:}
Aquellas \emph{Presentaciones} que cumplan las siguientes condiciones \textbf{NO} podrán ser eliminadas:
\begin{itemize}
\item {} 
Esten asociadas a un medicamento.

\end{itemize}

El sistema se encargará de informar al usuario las razones por las cuales la \emph{Presentación} seleccionada no puede eliminarse. En dicho caso, el sistema mostrara una ventana emergente (modal) como esta:

\noindent{\hspace*{\fill}\sphinxincludegraphics{{fallaeliminarpres}.png}\hspace*{\fill}}
\end{notice}


\subsection{Formulario de Búsqueda}
\label{presentaciones:formulario-de-busqueda}\label{presentaciones:formulario-busqueda-presentacion}
Si el usuario desea visualizar sólo aquellas \emph{Presentaciones} que cumplan con algunos criterios en específico, deberá utilizar el formulario de búsqueda.

\noindent{\hspace*{\fill}\sphinxincludegraphics{{busquedapres}.png}\hspace*{\fill}}

Este formulario sólo cuenta con la opción de búsqueda simple en base a la descripción de la \emph{Presentación}.

\begin{notice}{note}{Nota:}
Este campo es opcional, de no especificarse ningún criterio de búsqueda el sistema mostrará todas las \emph{Presentaciones}.
\end{notice}

El usuario tendrá que ingresar los parámetros de búsqueda en el formulario, y presionar el botón \sphinxcode{Buscar}. El sistema visualizará aquellas \emph{Presentaciones} que cumplan con todas las condiciones especificadas.


\section{Nombres Fantasia}
\label{nombresfantasia:nombres-fantasia}\label{nombresfantasia::doc}
Se presentará una pantalla que contendrá un listado con todos los \emph{Nombres Fantasía} que se encuentren registrados en el sistema hasta la fecha.

\noindent{\hspace*{\fill}\sphinxincludegraphics{{nombresfantasia}.png}\hspace*{\fill}}

Junto con el listado, se ofrecerán un conjunto de funcionalidades que permitirán manipular estos \emph{Nombres Fantasía}

Estas funcionalidades son:
\begin{itemize}
\item {} 
{\hyperref[nombresfantasia:alta\string-nombre\string-fantasia]{\sphinxcrossref{\DUrole{std,std-ref}{Alta Nombre Fantasía}}}}

\item {} 
{\hyperref[nombresfantasia:modificar\string-nombre\string-fantasia]{\sphinxcrossref{\DUrole{std,std-ref}{Modificar Nombre Fantasía}}}}

\item {} 
{\hyperref[nombresfantasia:eliminar\string-nombre\string-fantasia]{\sphinxcrossref{\DUrole{std,std-ref}{Eliminar Nombre Fantasía}}}}

\item {} 
{\hyperref[nombresfantasia:formulario\string-busqueda\string-nombre\string-fantasia]{\sphinxcrossref{\DUrole{std,std-ref}{Formulario de Búsqueda}}}}

\end{itemize}


\subsection{Alta Nombre Fantasía}
\label{nombresfantasia:alta-nombre-fantasia}\label{nombresfantasia:id1}
Si el usuario desea crear un nuevo \emph{Nombre Fantasía}, deberá presionar el botón \sphinxcode{Alta}.

\noindent{\hspace*{\fill}\sphinxincludegraphics{{btnaltanf}.png}\hspace*{\fill}}

A continuación el sistema lo redirigirá a la siguiente pantalla:

\noindent{\hspace*{\fill}\sphinxincludegraphics{{altanf}.png}\hspace*{\fill}}

En esta parte el usuario se le presentará un formulario y deberá ingresar los datos solicitados para dar de alta un nuevo \emph{Nombre Fantasía}.

\begin{notice}{attention}{Atención:}
El sistema siempre validará que la información ingresada sea correcta. En caso de que los datos ingresados sean incorrectos el sistema lo informará.
En este punto, las posibles causas de errores son:
\begin{itemize}
\item {} 
No se ingresó un nombre fantasía.

\item {} 
El nombre fantasía ingresado ya existe en el sistema.

\end{itemize}
\end{notice}

Una vez completado el formulario, el usuario tendrá dos opciones:
\begin{itemize}
\item {} 
Presionar el botón \sphinxcode{Guardar y Volver}.

\item {} 
Presionar el botón \sphinxcode{Guardar y Continuar}.

\end{itemize}

El botón \sphinxcode{Guardar y Volver} permite guardar el \emph{Nombre Fantasía} en el sistema y volver a la pantalla
principal de \emph{Nombres Fantasía}.

El botón \sphinxcode{Guardar y Continuar} permite guardar el \emph{Nombre Fantasía} en el sistema y seguir dando de alta nuevos \emph{Nombres Fantasía}.


\subsection{Modificar Nombre Fantasía}
\label{nombresfantasia:id2}\label{nombresfantasia:modificar-nombre-fantasia}
Si el usuario desea modificar los datos de un \emph{Nombre Fantasía}, deberá seleccionar el botón de \textbf{Acción} asociado al \emph{Nombre Fantasía} y presionar la pestaña \sphinxcode{Modificar}.

\noindent{\hspace*{\fill}\sphinxincludegraphics{{btnmodificarnf}.png}\hspace*{\fill}}

Una vez realizado el paso anterior, el sistema lo redirigirá a la siguiente pantalla:

\noindent{\hspace*{\fill}\sphinxincludegraphics{{modificarnf}.png}\hspace*{\fill}}

En esta parte al usuario se le presentará un formulario y deberá actualizar los datos asociados al \emph{Nombre Fantasía}.

\begin{notice}{attention}{Atención:}
El sistema siempre validará que la información ingresada sea correcta. En caso de que los datos ingresados sean incorrectos el sistema lo informará.
En este punto, las posibles causas de errores son:
\begin{itemize}
\item {} 
No se ingresó un nombre fantasía.

\item {} 
El nombre fantasía ingresado ya existe en el sistema.

\end{itemize}
\end{notice}

Una vez completado el formulario, el usuario deberá presionar el botón \sphinxcode{Guardar Cambios} y el sistema se encargara de actualizar los datos del \emph{Nombre Fantasía} seleccionado.


\subsection{Eliminar Nombre Fantasía}
\label{nombresfantasia:eliminar-nombre-fantasia}\label{nombresfantasia:id3}
Si el usuario desea eliminar un \emph{Nombre Fantasía}, deberá seleccionar el botón de \textbf{Acción} asociado al \emph{Nombre Fantasía} y presionar la pestaña \sphinxcode{Eliminar}.

\noindent{\hspace*{\fill}\sphinxincludegraphics{{btneliminarnf}.png}\hspace*{\fill}}

Una vez realizado el paso anterior aparecerá la siguiente ventana emergente (modal):

\noindent{\hspace*{\fill}\sphinxincludegraphics{{eliminarnf}.png}\hspace*{\fill}}

En esta parte el usuario deberá decidir si confirma la eliminación del \emph{Nombre Fantasía} o no. Si desea confirmar la eliminación deberá presionar el botón \sphinxcode{Confirmar}, caso contrario, presionará el botón \sphinxcode{Cancelar}.

\begin{notice}{note}{Nota:}\begin{quote}

Aquellos \emph{Nombres Fantasía} que cumplan las siguientes condiciones \textbf{NO} podrán ser eliminadas:
\begin{itemize}
\item {} 
Esten asociadas a un medicamento.

\end{itemize}

El sistema se encargará de informar al usuario las razones por las cuales el \emph{Nombre Fantasía} seleccionado no puede eliminarse. En dicho caso, el sistema mostrara una ventana emergente (modal) como esta:
\end{quote}

\noindent{\hspace*{\fill}\sphinxincludegraphics{{fallaeliminarnf}.png}\hspace*{\fill}}
\end{notice}


\subsection{Formulario de Búsqueda}
\label{nombresfantasia:formulario-busqueda-nombre-fantasia}\label{nombresfantasia:formulario-de-busqueda}
Si el usuario desea visualizar sólo aquellos \emph{Nombre Fantasía} que cumplan con algunos criterios en específico, deberá utilizar el formulario de búsqueda.

\noindent{\hspace*{\fill}\sphinxincludegraphics{{busquedanf}.png}\hspace*{\fill}}

Este formulario sólo cuenta con la opción de búsqueda simple en base al nombre del \emph{Nombre Fantasía}.

\begin{notice}{note}{Nota:}
Este campo es opcional, de no especificarse ningún criterio de búsqueda el sistema mostrará todos los \emph{Nombres Fantasía}.
\end{notice}

El usuario tendrá que ingresar los parámetros de búsqueda en el formulario, y presionar el botón \sphinxcode{Buscar}. El sistema visualizará aquellos \emph{Nombres Fantasía} que cumplan con todas las condiciones especificadas.


\chapter{Organizaciones}
\label{organizaciones::doc}\label{organizaciones:organizaciones}
Luego de hacer “click” en  la leyenda ``Organizaciones'', el sistema muestra un submenú donde el usuario puede seleccionar la organizacion sobre la cual realizar acciones.


\section{Farmacias}
\label{farmacias:farmacias}\label{farmacias::doc}
Se presentará una pantalla que contendrá un listado con todas las \emph{Farmacias} que se encuentren registradas en el sistema hasta la fecha.

\noindent{\hspace*{\fill}\sphinxincludegraphics{{farmacias}.png}\hspace*{\fill}}

Junto con el listado, se presentarán un conjunto de funcionalidades que permitirán manipular estas \emph{Farmacias}.

Estas funcionalidades son:
\begin{itemize}
\item {} 
{\hyperref[farmacias:alta\string-farmacia]{\sphinxcrossref{\DUrole{std,std-ref}{Alta Farmacia}}}}

\item {} 
{\hyperref[farmacias:modificar\string-farmacia]{\sphinxcrossref{\DUrole{std,std-ref}{Modificar Farmacia}}}}

\item {} 
{\hyperref[farmacias:eliminar\string-farmacia]{\sphinxcrossref{\DUrole{std,std-ref}{Eliminar Farmacia}}}}

\item {} 
{\hyperref[farmacias:formulario\string-busqueda\string-farmacia]{\sphinxcrossref{\DUrole{std,std-ref}{Formulario de Búsqueda}}}}

\end{itemize}


\subsection{Alta Farmacia}
\label{farmacias:alta-farmacia}\label{farmacias:id1}
Si el usuario desea crear una nueva \emph{Farmacia}, deberá presionar el botón \sphinxcode{Alta}.

\noindent{\hspace*{\fill}\sphinxincludegraphics{{btnaltafarm}.png}\hspace*{\fill}}

A continuación el sistema lo redirigirá a la siguiente pantalla:

\noindent{\hspace*{\fill}\sphinxincludegraphics{{altafarm}.png}\hspace*{\fill}}

En esta parte el usuario se le presentará un formulario y deberá ingresar los datos solicitados para dar de alta una nueva \emph{Farmacia}.

\begin{notice}{attention}{Atención:}
El sistema siempre validará que la información ingresada sea correcta. En caso de que los datos ingresados sean incorrectos el sistema lo informará.
En este punto, las posibles causas de errores son:
\begin{itemize}
\item {} 
Uno o más campos obligatorios vacíos.

\item {} 
Uno o más campos con un formato incorrecto.

\item {} 
El CUIT ingresado ya se encuentra asociado a otra organización.

\end{itemize}
\end{notice}

Una vez completado el formulario, el usuario tendrá dos opciones:
\begin{itemize}
\item {} 
Presionar el botón \sphinxcode{Guardar y Volver}.

\item {} 
Presionar el botón \sphinxcode{Guardar y Continuar}.

\end{itemize}

El botón \sphinxcode{Guardar y Volver} permite guardar la \emph{Farmacia} en el sistema y volver a la pantalla
principal de \emph{Farmacias}..

El botón \sphinxcode{Guardar y Continuar} permite guardar la \emph{Farmacia} en el sistema y seguir dando de alta nuevas \emph{Farmacias}.


\subsection{Modificar Farmacia}
\label{farmacias:modificar-farmacia}\label{farmacias:id2}
Si el usuario desea modificar los datos de una \emph{Farmacia}, deberá seleccionar el botón de \textbf{Acción} asociado a la \emph{Farmacia} y presionar la pestaña \sphinxcode{Modificar}.

\noindent{\hspace*{\fill}\sphinxincludegraphics{{btnmodificarfarm}.png}\hspace*{\fill}}

Una vez realizado el paso anterior, el sistema lo redirigirá a la siguiente pantalla:

\noindent{\hspace*{\fill}\sphinxincludegraphics{{modificarfarm}.png}\hspace*{\fill}}

En esta parte al usuario se le presentará un formulario y deberá actualizar los datos asociados a la \emph{Farmacia}.

\begin{notice}{attention}{Atención:}
El sistema siempre validará que la información ingresada sea correcta. En caso de que los datos ingresados sean incorrectos el sistema lo informará.
En este punto, las posibles causas de errores son:
\begin{itemize}
\item {} 
Uno o más campos obligatorios vacíos.

\item {} 
Uno o más campos con un formato incorrecto.

\end{itemize}
\end{notice}

Una vez completado el formulario, el usuario deberá presionar el botón \sphinxcode{Guardar Cambios} y el sistema se encargara de actualizar los datos de la \emph{Farmacia} seleccionada.


\subsection{Eliminar Farmacia}
\label{farmacias:eliminar-farmacia}\label{farmacias:id3}
Si el usuario desea eliminar una \emph{Farmacia}, deberá seleccionar el botón de \textbf{Acción} asociado a la \emph{Farmacia} y presionar la pestaña \sphinxcode{Eliminar}.

\noindent{\hspace*{\fill}\sphinxincludegraphics{{btneliminarfarm}.png}\hspace*{\fill}}

Una vez realizado el paso anterior aparecerá la siguiente ventana emergente (modal):

\noindent{\hspace*{\fill}\sphinxincludegraphics{{eliminarfarm}.png}\hspace*{\fill}}

En esta parte el usuario deberá decidir si confirma la eliminación de la \emph{Farmacia} o no. Si desea confirmar la eliminación deberá presionar el botón \sphinxcode{Confirmar}, caso contrario, presionará el botón \sphinxcode{Cancelar}.

\begin{notice}{note}{Nota:}
Aquellas \emph{Farmacias} que cumplan las siguientes condiciones \textbf{NO} podrán ser eliminadas:
\begin{itemize}
\item {} 
Esten asociadas a un Pedido de Farmacia que aún no ha sido completamente enviado.

\end{itemize}

El sistema se encargará de informar al usuario las razones por las cuales la \emph{Farmacia} seleccionado no puede eliminarse. En dicho caso, el sistema mostrara una ventana emergente (modal) como esta:

\noindent{\hspace*{\fill}\sphinxincludegraphics{{fallaeliminarfarm}.png}\hspace*{\fill}}
\end{notice}


\subsection{Formulario de Búsqueda}
\label{farmacias:formulario-de-busqueda}\label{farmacias:formulario-busqueda-farmacia}
Si el usuario desea visualizar sólo aquellas \emph{Farmacias} que cumplan con algunos criterios en específico, deberá utilizar el formulario de búsqueda.

\noindent{\hspace*{\fill}\sphinxincludegraphics{{busquedafarm}.png}\hspace*{\fill}}

Este formulario cuenta con dos modalidades:
\begin{itemize}
\item {} 
Búsqueda simple: permite buscar las \emph{Farmacias} por razon social.

\item {} 
Búsqueda avanzada: permite buscar las \emph{Farmacias} por razon social, localidad.

\end{itemize}

\begin{notice}{note}{Nota:}
Todos los campos son opcionales, de no especificarse ningún criterio de búsqueda el sistema mostrará todas las \emph{Farmacias}.
\end{notice}

El usuario tendrá que ingresar los parámetros de búsqueda en el formulario, y presionar el botón \sphinxcode{Buscar}. El sistema visualizará aquellas \emph{Farmacias} que cumplan con todas las condiciones especificadas.

Si el usuario desea limpiar los filtros activos, deberá presionar el boton \sphinxcode{Limpiar}.

\noindent{\hspace*{\fill}\sphinxincludegraphics{{limpiarfarm}.png}\hspace*{\fill}}


\section{Clínicas}
\label{clinicas:clinicas}\label{clinicas::doc}
Se presentará una pantalla que contendrá un listado con todas las \emph{Clínicas} que se encuentren registradas en el sistema hasta la fecha.

\noindent{\hspace*{\fill}\sphinxincludegraphics{{clinicas}.png}\hspace*{\fill}}

Junto con el listado, se presentarán un conjunto de funcionalidades que permitirán manipular estas \emph{Clínicas}.

Estas funcionalidades son:
\begin{itemize}
\item {} 
{\hyperref[clinicas:alta\string-clinica]{\sphinxcrossref{\DUrole{std,std-ref}{Alta Clínica}}}}

\item {} 
{\hyperref[clinicas:ver\string-obras\string-sociales]{\sphinxcrossref{\DUrole{std,std-ref}{Ver Obras Sociales}}}}

\item {} 
{\hyperref[clinicas:modificar\string-clinica]{\sphinxcrossref{\DUrole{std,std-ref}{Modificar Clínica}}}}

\item {} 
{\hyperref[clinicas:eliminar\string-clinica]{\sphinxcrossref{\DUrole{std,std-ref}{Eliminar Clínica}}}}

\item {} 
{\hyperref[clinicas:formulario\string-busqueda\string-clinica]{\sphinxcrossref{\DUrole{std,std-ref}{Formulario de Búsqueda}}}}

\end{itemize}


\subsection{Alta Clínica}
\label{clinicas:alta-clinica}\label{clinicas:id1}
Si el usuario desea crear una nueva \emph{Clínica}, deberá presionar el botón \sphinxcode{Alta}.

\noindent{\hspace*{\fill}\sphinxincludegraphics{{btnaltaclin}.png}\hspace*{\fill}}

A continuación el sistema lo redirigirá a la siguiente pantalla:

\noindent{\hspace*{\fill}\sphinxincludegraphics{{altaclin}.png}\hspace*{\fill}}

En esta parte el usuario se le presentará un formulario y deberá ingresar los datos solicitados para dar de alta una nueva \emph{Clínica}.

\begin{notice}{attention}{Atención:}
El sistema siempre validará que la información ingresada sea correcta. En caso de que los datos ingresados sean incorrectos el sistema lo informará.
En este punto, las posibles causas de errores son:
\begin{itemize}
\item {} 
Uno o más campos obligatorios vacíos.

\item {} 
Uno o más campos con un formato incorrecto.

\item {} 
El CUIT ingresado ya se encuentra asociado a otra organización.

\end{itemize}
\end{notice}

Una vez completado el formulario, el usuario tendrá dos opciones:
\begin{itemize}
\item {} 
Presionar el botón \sphinxcode{Guardar y Volver}.

\item {} 
Presionar el botón \sphinxcode{Guardar y Continuar}.

\end{itemize}

El botón \sphinxcode{Guardar y Volver} permite guardar la \emph{Clínica} en el sistema y volver a la pantalla
principal de \emph{Clínicas}..

El botón \sphinxcode{Guardar y Continuar} permite guardar la \emph{Clínica} en el sistema y seguir dando de alta nuevas \emph{Clínicas}.


\subsection{Ver Obras Sociales}
\label{clinicas:ver-obras-sociales}\label{clinicas:id2}
Si el usuario desea ver las obras sociales asociadas a una \emph{Clínica}, deberá seleccionar el botón de \textbf{Acción} asociado a la \emph{Clínica} y presionar la pestaña \sphinxcode{Ver Obras Sociales}.

\noindent{\hspace*{\fill}\sphinxincludegraphics{{btnobrasclin}.png}\hspace*{\fill}}

Una vez realizado el paso anterior aparecerá en pantalla el listado con las Obras Sociales que posee:

\noindent{\hspace*{\fill}\sphinxincludegraphics{{obrasclin}.png}\hspace*{\fill}}

Esta ventana mostrará todas los obras sociales vinculadas a la \emph{Clínica} seleccionada.


\subsection{Modificar Clínica}
\label{clinicas:modificar-clinica}\label{clinicas:id3}
Si el usuario desea modificar los datos de una \emph{Clínica}, deberá seleccionar el botón de \textbf{Acción} asociado a la \emph{Clínica} y presionar la pestaña \sphinxcode{Modificar}.

\noindent{\hspace*{\fill}\sphinxincludegraphics{{btnmodificarclin}.png}\hspace*{\fill}}

Una vez realizado el paso anterior, el sistema lo redirigirá a la siguiente pantalla:

\noindent{\hspace*{\fill}\sphinxincludegraphics{{modificarclin}.png}\hspace*{\fill}}

En esta parte al usuario se le presentará un formulario y deberá actualizar los datos asociados a la \emph{Clínica}.

\begin{notice}{attention}{Atención:}
El sistema siempre validará que la información ingresada sea correcta. En caso de que los datos ingresados sean incorrectos el sistema lo informará.
En este punto, las posibles causas de errores son:
\begin{itemize}
\item {} 
Uno o más campos obligatorios vacíos.

\item {} 
Uno o más campos con un formato incorrecto.

\end{itemize}
\end{notice}

Una vez completado el formulario, el usuario deberá presionar el botón \sphinxcode{Guardar Cambios} y el sistema se encargara de actualizar los datos de la \emph{Clínica} seleccionada.


\subsection{Eliminar Clínica}
\label{clinicas:id4}\label{clinicas:eliminar-clinica}
Si el usuario desea eliminar una \emph{Clínica}, deberá seleccionar el botón de \textbf{Acción} asociado a la \emph{Clínica} y presionar la pestaña \sphinxcode{Eliminar}.

\noindent{\hspace*{\fill}\sphinxincludegraphics{{btneliminarclin}.png}\hspace*{\fill}}

Una vez realizado el paso anterior aparecerá la siguiente ventana emergente (modal):

\noindent{\hspace*{\fill}\sphinxincludegraphics{{eliminarclin}.png}\hspace*{\fill}}

En esta parte el usuario deberá decidir si confirma la eliminación de la \emph{Clínica} o no. Si desea confirmar la eliminación deberá presionar el botón \sphinxcode{Confirmar}, caso contrario, presionará el botón \sphinxcode{Cancelar}.


\subsection{Formulario de Búsqueda}
\label{clinicas:formulario-busqueda-clinica}\label{clinicas:formulario-de-busqueda}
Si el usuario desea visualizar sólo aquellas \emph{Clínicas} que cumplan con algunos criterios en específico, deberá utilizar el formulario de búsqueda.

\noindent{\hspace*{\fill}\sphinxincludegraphics{{busquedaclin}.png}\hspace*{\fill}}

Este formulario cuenta con dos modalidades:
\begin{itemize}
\item {} 
Búsqueda simple: permite buscar las \emph{Clínicas} por razon social.

\item {} 
Búsqueda avanzada: permite buscar las \emph{Clínicas} por razon social, localidad, obra social.

\end{itemize}

\begin{notice}{note}{Nota:}
Todos los campos son opcionales, de no especificarse ningún criterio de búsqueda el sistema mostrará todas las \emph{Clínicas}.
\end{notice}

El usuario tendrá que ingresar los parámetros de búsqueda en el formulario, y presionar el botón \sphinxcode{Buscar}. El sistema visualizará aquellas \emph{Clínicas} que cumplan con todas las condiciones especificadas.

Si el usuario desea limpiar los filtros activos, deberá presionar el boton \sphinxcode{Limpiar}.

\noindent{\hspace*{\fill}\sphinxincludegraphics{{limpiarclin}.png}\hspace*{\fill}}


\section{Laboratorios}
\label{laboratorios::doc}\label{laboratorios:laboratorios}
Se presentará una pantalla que contendrá un listado con todos los \emph{Laboratorios} que se encuentren registradas en el sistema hasta la fecha.

\noindent{\hspace*{\fill}\sphinxincludegraphics{{laboratorios}.png}\hspace*{\fill}}

Junto con el listado, se presentarán un conjunto de funcionalidades que permitirán manipular estos \emph{Laboratorios}.

Estas funcionalidades son:
\begin{itemize}
\item {} 
{\hyperref[laboratorios:alta\string-laboratorio]{\sphinxcrossref{\DUrole{std,std-ref}{Alta Laboratorio}}}}

\item {} 
{\hyperref[laboratorios:modificar\string-laboratorio]{\sphinxcrossref{\DUrole{std,std-ref}{Modificar Laboratorio}}}}

\item {} 
{\hyperref[laboratorios:eliminar\string-laboratorio]{\sphinxcrossref{\DUrole{std,std-ref}{Eliminar Laboratorio}}}}

\item {} 
{\hyperref[laboratorios:formulario\string-busqueda\string-laboratorio]{\sphinxcrossref{\DUrole{std,std-ref}{Formulario de Búsqueda}}}}

\end{itemize}


\subsection{Alta Laboratorio}
\label{laboratorios:alta-laboratorio}\label{laboratorios:id1}
Si el usuario desea crear un nuevo \emph{Laboratorio}, deberá presionar el botón \sphinxcode{Alta}.

\noindent{\hspace*{\fill}\sphinxincludegraphics{{btnaltalab}.png}\hspace*{\fill}}

A continuación el sistema lo redirigirá a la siguiente pantalla:

\noindent{\hspace*{\fill}\sphinxincludegraphics{{altalab}.png}\hspace*{\fill}}

En esta parte el usuario se le presentará un formulario y deberá ingresar los datos solicitados para dar de alta un nuevo \emph{Laboratorio}.

\begin{notice}{attention}{Atención:}
El sistema siempre validará que la información ingresada sea correcta. En caso de que los datos ingresados sean incorrectos el sistema lo informará.
En este punto, las posibles causas de errores son:
\begin{itemize}
\item {} 
Uno o más campos obligatorios vacíos.

\item {} 
Uno o más campos con un formato incorrecto.

\item {} 
El CUIT ingresado ya se encuentra asociado a otra organización.

\end{itemize}
\end{notice}

Una vez completado el formulario, el usuario tendrá dos opciones:
\begin{itemize}
\item {} 
Presionar el botón \sphinxcode{Guardar y Volver}.

\item {} 
Presionar el botón \sphinxcode{Guardar y Continuar}.

\end{itemize}

El botón \sphinxcode{Guardar y Volver} permite guardar el \emph{Laboratorio} en el sistema y volver a la pantalla
principal de \emph{Laboratorios}..

El botón \sphinxcode{Guardar y Continuar} permite guardar el \emph{Laboratorio} en el sistema y seguir dando de alta nuevos \emph{Laboratorios}.


\subsection{Modificar Laboratorio}
\label{laboratorios:id2}\label{laboratorios:modificar-laboratorio}
Si el usuario desea modificar los datos de un \emph{Laboratorio}, deberá seleccionar el botón de \textbf{Acción} asociado al \emph{Laboratorio} y presionar la pestaña \sphinxcode{Modificar}.

\noindent{\hspace*{\fill}\sphinxincludegraphics{{btnmodificarlab}.png}\hspace*{\fill}}

Una vez realizado el paso anterior, el sistema lo redirigirá a la siguiente pantalla:

\noindent{\hspace*{\fill}\sphinxincludegraphics{{modificarlab}.png}\hspace*{\fill}}

En esta parte al usuario se le presentará un formulario y deberá actualizar los datos asociados al \emph{Laboratorio}.

\begin{notice}{attention}{Atención:}
El sistema siempre validará que la información ingresada sea correcta. En caso de que los datos ingresados sean incorrectos el sistema lo informará.
En este punto, las posibles causas de errores son:
\begin{itemize}
\item {} 
Uno o más campos obligatorios vacíos.

\item {} 
Uno o más campos con un formato incorrecto.

\end{itemize}
\end{notice}

Una vez completado el formulario, el usuario deberá presionar el botón \sphinxcode{Guardar Cambios} y el sistema se encargara de actualizar los datos del \emph{Laboratorio} seleccionado.


\subsection{Eliminar Laboratorio}
\label{laboratorios:eliminar-laboratorio}\label{laboratorios:id3}
Si el usuario desea eliminar un \emph{Laboratorio}, deberá seleccionar el botón de \textbf{Acción} asociado al \emph{Laboratorio} y presionar la pestaña \sphinxcode{Eliminar}.

\noindent{\hspace*{\fill}\sphinxincludegraphics{{btneliminarlab}.png}\hspace*{\fill}}

Una vez realizado el paso anterior aparecerá la siguiente ventana emergente (modal):

\noindent{\hspace*{\fill}\sphinxincludegraphics{{eliminarlab}.png}\hspace*{\fill}}

En esta parte el usuario deberá decidir si confirma la eliminación del \emph{Laboratorio} o no. Si desea confirmar la eliminación deberá presionar el botón \sphinxcode{Confirmar}, caso contrario, presionará el botón \sphinxcode{Cancelar}.

\begin{notice}{note}{Nota:}
Aquellos \emph{Laboratorios} que cumplan las siguientes condiciones \textbf{NO} podrán ser eliminadas:
\begin{itemize}
\item {} 
Esten asociados a un Pedido a Laboratorio que aún no ha sido completamente recepcionado

\item {} 
Esten asociados a un medicamento que se encuentre en un detalle de un Pedido de Farmacia que aún no haya sido completamente enviado.

\item {} 
Esten asociados a un medicamento que posee stock.

\end{itemize}

El sistema se encargará de informar al usuario las razones por las cuales el \emph{Laboratorio} seleccionado no puede eliminarse. En dicho caso, el sistema mostrara una ventana emergente (modal) como esta:

\noindent{\hspace*{\fill}\sphinxincludegraphics{{fallaeliminarlab}.png}\hspace*{\fill}}
\end{notice}


\subsection{Formulario de Búsqueda}
\label{laboratorios:formulario-de-busqueda}\label{laboratorios:formulario-busqueda-laboratorio}
Si el usuario desea visualizar sólo aquellos \emph{Laboratorios} que cumplan con algunos criterios en específico, deberá utilizar el formulario de búsqueda.

\noindent{\hspace*{\fill}\sphinxincludegraphics{{busquedalab}.png}\hspace*{\fill}}

Este formulario cuenta con dos modalidades:
\begin{itemize}
\item {} 
Búsqueda simple: permite buscar los \emph{Laboratorios} por razon social.

\item {} 
Búsqueda avanzada: permite buscar los \emph{Laboratorios} por razon social, localidad.

\end{itemize}

\begin{notice}{note}{Nota:}
Todos los campos son opcionales, de no especificarse ningún criterio de búsqueda el sistema mostrará todos los \emph{Laboratorios}.
\end{notice}

El usuario tendrá que ingresar los parámetros de búsqueda en el formulario, y presionar el botón \sphinxcode{Buscar}. El sistema visualizará aquellos \emph{Laboratorios} que cumplan con todas las condiciones especificadas.

Si el usuario desea limpiar los filtros activos, deberá presionar el boton \sphinxcode{Limpiar}.

\noindent{\hspace*{\fill}\sphinxincludegraphics{{limpiarlab}.png}\hspace*{\fill}}


\section{Obras Sociales}
\label{obrassociales::doc}\label{obrassociales:obras-sociales}
Se presentará una pantalla que contendrá un listado con todas las \emph{Obras Sociales} que se encuentren registradas en el sistema hasta la fecha.

\noindent{\hspace*{\fill}\sphinxincludegraphics{{obrassociales}.png}\hspace*{\fill}}

Junto con el listado, se presentarán un conjunto de funcionalidades que permitirán manipular estas \emph{Obras Sociales}.

Estas funcionalidades son:
\begin{itemize}
\item {} 
{\hyperref[obrassociales:alta\string-obrasocial]{\sphinxcrossref{\DUrole{std,std-ref}{Alta Obra Social}}}}

\item {} 
{\hyperref[obrassociales:modificar\string-obrasocial]{\sphinxcrossref{\DUrole{std,std-ref}{Modificar Obra Social}}}}

\item {} 
{\hyperref[obrassociales:eliminar\string-obrasocial]{\sphinxcrossref{\DUrole{std,std-ref}{Eliminar Obra Social}}}}

\item {} 
{\hyperref[obrassociales:formulario\string-busqueda\string-obrasocial]{\sphinxcrossref{\DUrole{std,std-ref}{Formulario de Búsqueda}}}}

\end{itemize}


\subsection{Alta Obra Social}
\label{obrassociales:alta-obrasocial}\label{obrassociales:alta-obra-social}
Si el usuario desea crear una nueva \emph{Obra Social}, deberá presionar el botón \sphinxcode{Alta}.

\noindent{\hspace*{\fill}\sphinxincludegraphics{{btnaltaobrasocial}.png}\hspace*{\fill}}

A continuación el sistema lo redirigirá a la siguiente pantalla:

\noindent{\hspace*{\fill}\sphinxincludegraphics{{altaobrasocial}.png}\hspace*{\fill}}

En esta parte el usuario se le presentará un formulario y deberá ingresar los datos solicitados para dar de alta una nueva \emph{Obra Social}.

\begin{notice}{attention}{Atención:}
El sistema siempre validará que la información ingresada sea correcta. En caso de que los datos ingresados sean incorrectos el sistema lo informará.
En este punto, las posibles causas de errores son:
\begin{itemize}
\item {} 
Uno o más campos obligatorios vacíos.

\item {} 
Uno o más campos con un formato incorrecto.

\item {} 
El CUIT ingresado ya se encuentra asociado a otra organización.

\end{itemize}
\end{notice}

Una vez completado el formulario, el usuario tendrá dos opciones:
\begin{itemize}
\item {} 
Presionar el botón \sphinxcode{Guardar y Volver}.

\item {} 
Presionar el botón \sphinxcode{Guardar y Continuar}.

\end{itemize}

El botón \sphinxcode{Guardar y Volver} permite guardar la \emph{Obra Social} en el sistema y volver a la pantalla
principal de \emph{Obra Social}..

El botón \sphinxcode{Guardar y Continuar} permite guardar la \emph{Obra Social} en el sistema y seguir dando de alta nuevas \emph{Obra Social}.


\subsection{Modificar Obra Social}
\label{obrassociales:modificar-obra-social}\label{obrassociales:modificar-obrasocial}
Si el usuario desea modificar los datos de una \emph{Obra Social}, deberá seleccionar el botón de \textbf{Acción} asociado a la \emph{Obra Social} y presionar la pestaña \sphinxcode{Modificar}.

\noindent{\hspace*{\fill}\sphinxincludegraphics{{btnmodificarfarm}.png}\hspace*{\fill}}

Una vez realizado el paso anterior, el sistema lo redirigirá a la siguiente pantalla:

\noindent{\hspace*{\fill}\sphinxincludegraphics{{modificarfarm}.png}\hspace*{\fill}}

En esta parte al usuario se le presentará un formulario y deberá actualizar los datos asociados a la \emph{Obra Social}.

\begin{notice}{attention}{Atención:}
El sistema siempre validará que la información ingresada sea correcta. En caso de que los datos ingresados sean incorrectos el sistema lo informará.
En este punto, las posibles causas de errores son:
\begin{itemize}
\item {} 
Uno o más campos obligatorios vacíos.

\item {} 
Uno o más campos con un formato incorrecto.

\end{itemize}
\end{notice}

Una vez completado el formulario, el usuario deberá presionar el botón \sphinxcode{Guardar Cambios} y el sistema se encargara de actualizar los datos de la \emph{Obra Social} seleccionada.


\subsection{Eliminar Obra Social}
\label{obrassociales:eliminar-obrasocial}\label{obrassociales:eliminar-obra-social}
Si el usuario desea eliminar una \emph{Obra Social}, deberá seleccionar el botón de \textbf{Acción} asociado a la \emph{Obra Social} y presionar la pestaña \sphinxcode{Eliminar}.

\noindent{\hspace*{\fill}\sphinxincludegraphics{{btneliminarfarm}.png}\hspace*{\fill}}

Una vez realizado el paso anterior aparecerá la siguiente ventana emergente (modal):

\noindent{\hspace*{\fill}\sphinxincludegraphics{{eliminarfarm}.png}\hspace*{\fill}}

En esta parte el usuario deberá decidir si confirma la eliminación de la \emph{Obra Social} o no. Si desea confirmar la eliminación deberá presionar el botón \sphinxcode{Confirmar}, caso contrario, presionará el botón \sphinxcode{Cancelar}.

\begin{notice}{note}{Nota:}
Aquellas \emph{Obras Sociales} que cumplan las siguientes condiciones \textbf{NO} podrán ser eliminadas:
\begin{itemize}
\item {} 
Esten asociadas a un Pedido de Obra Social que aún no ha sido completamente enviado.

\end{itemize}

El sistema se encargará de informar al usuario las razones por las cuales la \emph{Obra Social} seleccionada no puede eliminarse. En dicho caso, el sistema mostrara una ventana emergente (modal) como esta:

\noindent{\hspace*{\fill}\sphinxincludegraphics{{fallaeliminarfarm}.png}\hspace*{\fill}}
\end{notice}


\subsection{Formulario de Búsqueda}
\label{obrassociales:formulario-de-busqueda}\label{obrassociales:formulario-busqueda-obrasocial}
Si el usuario desea visualizar sólo aquellas \emph{Obras Sociales} que cumplan con algunos criterios en específico, deberá utilizar el formulario de búsqueda.

\noindent{\hspace*{\fill}\sphinxincludegraphics{{busquedafarm}.png}\hspace*{\fill}}

Este formulario cuenta con dos modalidades:
\begin{itemize}
\item {} 
Búsqueda simple: permite buscar las \emph{Obras Sociales} por razon social.

\item {} 
Búsqueda avanzada: permite buscar las \emph{Obras Sociales} por razon social, localidad.

\end{itemize}

\begin{notice}{note}{Nota:}
Todos los campos son opcionales, de no especificarse ningún criterio de búsqueda el sistema mostrará todas las \emph{Obras Sociales}.
\end{notice}

El usuario tendrá que ingresar los parámetros de búsqueda en el formulario, y presionar el botón \sphinxcode{Buscar}. El sistema visualizará aquellas \emph{Obras Sociales} que cumplan con todas las condiciones especificadas.

Si el usuario desea limpiar los filtros activos, deberá presionar el boton \sphinxcode{Limpiar}.

\noindent{\hspace*{\fill}\sphinxincludegraphics{{limpiarfarm}.png}\hspace*{\fill}}


\chapter{Pedidos}
\label{pedidos::doc}\label{pedidos:pedidos}
Luego de hacer “click” en  la leyenda “Pedidos”, el sistema muestra un submenú donde el usuario puede seleccionar la actividad que desea realizar.


\section{Pedidos de Farmacia}
\label{pedidosfarmacia::doc}\label{pedidosfarmacia:pedidos-de-farmacia}
Se presentará una pantalla que contendrá un listado con todos los \emph{Pedidos de Farmacia} que se encuentren registrados en el sistema hasta la fecha.

\noindent{\hspace*{\fill}\sphinxincludegraphics{{pedidosfarmacia}.png}\hspace*{\fill}}

Junto con el listado, se ofrecerán un conjunto de funcionalidades que permitirán manipular estos \emph{Pedidos de Farmacia}.
Estas funcionalidades son:
\begin{itemize}
\item {} 
{\hyperref[pedidosfarmacia:alta\string-pf]{\sphinxcrossref{\DUrole{std,std-ref}{Alta Pedido}}}}

\item {} 
{\hyperref[pedidosfarmacia:ver\string-detalles\string-pf]{\sphinxcrossref{\DUrole{std,std-ref}{Ver Detalles}}}}

\item {} 
{\hyperref[pedidosfarmacia:ver\string-remitos\string-pf]{\sphinxcrossref{\DUrole{std,std-ref}{Ver Remitos}}}}

\item {} 
{\hyperref[pedidosfarmacia:formulario\string-busqueda\string-pf]{\sphinxcrossref{\DUrole{std,std-ref}{Formulario de Búsqueda}}}}

\end{itemize}


\subsection{Alta Pedido}
\label{pedidosfarmacia:alta-pf}\label{pedidosfarmacia:alta-pedido}
Si el usuario desea crear un nuevo \emph{Pedido de Farmacia}, deberá presionar el botón \sphinxcode{Alta}.

\noindent{\hspace*{\fill}\sphinxincludegraphics{{btnaltapedfarm}.png}\hspace*{\fill}}

A continuación el sistema lo redirigirá a la siguiente pantalla:

\noindent{\hspace*{\fill}\sphinxincludegraphics{{altapedfarm}.png}\hspace*{\fill}}

En este punto el usuario deberá seleccionar la fecha en que llegó el pedido y la farmacia que lo realizó. A continuación deberá presionar el botón \sphinxcode{Crear Pedido}.

\begin{notice}{attention}{Atención:}
El sistema siempre validará que la información ingresada sea correcta. En caso de que los datos ingresados sean incorrectos el sistema lo informará.
En este punto, las posibles causas de errores son:
\begin{itemize}
\item {} 
La farmacia ingresada no existe.

\item {} 
La fecha no existe.

\item {} 
La fecha ingresada esta fuera del rango válido.

\end{itemize}
\end{notice}

Una vez presionado el botón \sphinxcode{Crear Pedido}, se mostrará la siguiente pantalla:

\noindent{\hspace*{\fill}\sphinxincludegraphics{{detallespedfarm}.png}\hspace*{\fill}}

Esta pantalla es la encargada de visualizar aquellos detalles que se irán asociando al \emph{Pedido de Farmacia}.
La misma ofrece las siguientes funcionalidades:
\begin{itemize}
\item {} 
{\hyperref[pedidosfarmacia:agregar\string-detalle\string-pf]{\sphinxcrossref{\DUrole{std,std-ref}{Agregar Detalle}}}}

\item {} 
{\hyperref[pedidosfarmacia:modificar\string-detalle\string-pf]{\sphinxcrossref{\DUrole{std,std-ref}{Modificar Detalle}}}}

\item {} 
{\hyperref[pedidosfarmacia:eliminar\string-detalle\string-pf]{\sphinxcrossref{\DUrole{std,std-ref}{Eliminar Detalle}}}}

\item {} 
{\hyperref[pedidosfarmacia:registrar\string-pedido\string-pf]{\sphinxcrossref{\DUrole{std,std-ref}{Registrar Pedido}}}}

\end{itemize}


\subsubsection{Agregar Detalle}
\label{pedidosfarmacia:agregar-detalle-pf}\label{pedidosfarmacia:agregar-detalle}
Si el usuario desea agregar un detalle al \emph{Pedido de Farmacia}, deberá presionar el botón \sphinxcode{Alta Detalle}.

\noindent{\hspace*{\fill}\sphinxincludegraphics{{btnadddetallepedfarm}.png}\hspace*{\fill}}

Una vez realizado el paso anterior aparecerá la siguiente ventana emergente (modal):

\noindent{\hspace*{\fill}\sphinxincludegraphics{{newdetallepedfarm}.png}\hspace*{\fill}}

En esta parte, se presentará un formulario que el usuario deberá completar para poder dar de alta un nuevo detalle.

\begin{notice}{attention}{Atención:}
El sistema siempre validará que la información ingresada sea correcta. En caso de que los datos ingresados sean incorrectos el sistema lo informará.
En este punto, las posibles causas de errores son:
\begin{itemize}
\item {} 
No se seleccionó un medicamento.

\item {} 
No se ingresó una cantidad.

\item {} 
La cantidad ingresada no posee un formato correcto.

\item {} 
La cantidad ingresada es menor a cero.

\end{itemize}
\end{notice}

Una vez completado el formulario, el usuario deberá presionar el botón \sphinxcode{Guardar} y el sistema se encargara de agregar el nuevo detalle al pedido.
El usuario podrá seguir dando de alta nuevos detalles, hasta donde considere necesario. Una vez que esto suceda deberá presionar el botón \sphinxcode{Cerrar} y la ventana emergente desaparecerá.


\subsubsection{Modificar Detalle}
\label{pedidosfarmacia:modificar-detalle-pf}\label{pedidosfarmacia:modificar-detalle}
Si el usuario desea modificar un detalle del \emph{Pedido de Farmacia}, deberá seleccionar el detalle que desea actualizar y presionar el botón \sphinxcode{Modificar Detalle}.

\noindent{\hspace*{\fill}\sphinxincludegraphics{{btnupddetallepedfarm}.png}\hspace*{\fill}}

Una vez realizado el paso anterior aparecerá la siguiente ventana emergente (modal):

\noindent{\hspace*{\fill}\sphinxincludegraphics{{upddetallepedfarm}.png}\hspace*{\fill}}

En esta parte, se presentará un formulario con la información actual del detalle y el usuario deberá actualizar aquella que considere necesaria.

\begin{notice}{attention}{Atención:}
El sistema siempre validará que la información ingresada sea correcta. En caso de que los datos ingresados sean incorrectos el sistema lo informará.
En este punto, las posibles causas de errores son:
\begin{itemize}
\item {} 
No se ingresó una cantidad.

\item {} 
La cantidad ingresada no posee un formato correcto.

\item {} 
La cantidad ingresada es menor a cero.

\end{itemize}
\end{notice}

Una vez completado el formulario, el usuario deberá presionar el botón \sphinxcode{Guardar} y el sistema se encargará de actualizar la información de dicho detalle.


\subsubsection{Eliminar Detalle}
\label{pedidosfarmacia:eliminar-detalle-pf}\label{pedidosfarmacia:eliminar-detalle}
Si el usuario desea eliminar un detalle del \emph{Pedido de Farmacia}, deberá seleccionar el detalle que desea eliminar y presionar el botón \sphinxcode{Baja Detalle}.

\noindent{\hspace*{\fill}\sphinxincludegraphics{{btndeldetallepedfarm}.png}\hspace*{\fill}}

Una vez realizado el paso anterior aparecerá la siguiente ventana emergente (modal):

\noindent{\hspace*{\fill}\sphinxincludegraphics{{deldetallepedfarm}.png}\hspace*{\fill}}

En esta parte el usuario deberá decidir si confirma la eliminación del detalle o no. Si desea confirmar la eliminación deberá presionar el botón \sphinxcode{Confirmar}, caso contrario, presionará el botón \sphinxcode{Cancelar}.


\subsubsection{Registrar Pedido}
\label{pedidosfarmacia:registrar-pedido}\label{pedidosfarmacia:registrar-pedido-pf}
Si el usuario desea registrar el \emph{Pedido de Farmacia}, deberá presionar el botón \sphinxcode{Registrar}.

\noindent{\hspace*{\fill}\sphinxincludegraphics{{btnregpedfarm}.png}\hspace*{\fill}}

\begin{notice}{attention}{Atención:}
El sistema siempre validará que la información del \emph{Pedido a de Farmacia} sea correcta. En caso de que esta información sea incorrecta el sistema lo informará.
En este punto, las posibles causas de errores son:
\begin{itemize}
\item {} 
El pedido no contiene detalles

\item {} 
El pedido ya ha sido registrado anteriormente

\end{itemize}
\end{notice}

Una vez presionado el botón \sphinxcode{Registrar}, el sistema se encargará de crear el \emph{Pedido de Farmacia} y se mostrará la siguiente ventana emergente (modal).

\noindent{\hspace*{\fill}\sphinxincludegraphics{{regpedfarm}.png}\hspace*{\fill}}


\subsection{Ver Detalles}
\label{pedidosfarmacia:ver-detalles}\label{pedidosfarmacia:ver-detalles-pf}
Si el usuario desea ver los detalles de un \emph{Pedido de Farmacia}, deberá seleccionar el botón de \textbf{Acción} asociado a dicho pedido y presionar la pestaña \sphinxcode{Ver Detalles}.

\noindent{\hspace*{\fill}\sphinxincludegraphics{{btndetallespedfarm}.png}\hspace*{\fill}}

Una vez realizado el paso anterior aparecerá la siguiente ventana emergente (modal):

\noindent{\hspace*{\fill}\sphinxincludegraphics{{verdetallespedfarm}.png}\hspace*{\fill}}

Esta ventana mostrará todos los detalles del \emph{Pedido de Farmacia} seleccionado.


\subsection{Ver Remitos}
\label{pedidosfarmacia:ver-remitos}\label{pedidosfarmacia:ver-remitos-pf}
Si el usuario desea ver los remitos asociados a un \emph{Pedido de Farmacia}, deberá seleccionar el botón de \textbf{Acción} asociado a dicho pedido y presionar la pestaña \sphinxcode{Ver Remitos}.

\noindent{\hspace*{\fill}\sphinxincludegraphics{{btnremitospedfarm}.png}\hspace*{\fill}}

Una vez realizado el paso anterior aparecerá la siguiente ventana emergente (modal):

\noindent{\hspace*{\fill}\sphinxincludegraphics{{remitospedfarm}.png}\hspace*{\fill}}

Esta ventana mostrará todos los remitos vinculados al \emph{Pedido de Farmacia} seleccionado.

\begin{notice}{note}{Nota:}
En caso de que el pedido no tenga remitos asociados el sistema lo informará.
\end{notice}

El usuario tendra la opción de visualizar un remito en PDF, presionanado el boton \sphinxcode{Descargar} asociado a él.

\noindent{\hspace*{\fill}\sphinxincludegraphics{{remitopedidofarmacia}.png}\hspace*{\fill}}

\textless{}\textless{}\textless{}\textless{}\textless{}\textless{}\textless{} HEAD
.. \_notific:

\noindent{\hspace*{\fill}\sphinxincludegraphics{{notificacionDePedidoMobile}.png}\hspace*{\fill}}

\noindent{\hspace*{\fill}\sphinxincludegraphics{{notificacionDePedidoMobileMini}.png}\hspace*{\fill}}

Al presionar en el boton \sphinxcode{Ver} se desplegaran los nuevos pedidos:

\noindent{\hspace*{\fill}\sphinxincludegraphics{{notificacionDePedidoMobileVer}.png}\hspace*{\fill}}

\textless{}\textless{}\textless{}\textless{}\textless{}\textless{}\textless{} HEAD


\subsection{Pedidos con Faltantes distribuidos}
\label{pedidosfarmacia:pedidos-con-faltantes-distribuidos}
Cuando una farmacia hace un pedido (pequeño de hasta 20 productos) y no tenemos stock en drogueria pero si en Farmacia, podemos optar por pedir a la farmacia el envio del faltante.
En este caso el pedido de ejemplo tenia un pendiente de 10.

\noindent{\hspace*{\fill}\sphinxincludegraphics{{pendientes1}.png}\hspace*{\fill}}

Para completarlo debemos presionar en el boton de \sphinxcode{Intentar completar con farmacias} y luego \sphinxcode{Busqueda en Farmacia}.

\noindent{\hspace*{\fill}\sphinxincludegraphics{{pendientes2}.png}\hspace*{\fill}}

Por defecto nos dara cual es el movimiento de medicamentos optimo. Pero si queremos podemos elejir de farmacias.

\noindent{\hspace*{\fill}\sphinxincludegraphics{{pendientes3-1}.png}\hspace*{\fill}}

Nos aparecera un cartel de confirmacion.

\noindent{\hspace*{\fill}\sphinxincludegraphics{{pendienteConfirm}.png}\hspace*{\fill}}

Sino la otra opcion es:

\noindent{\hspace*{\fill}\sphinxincludegraphics{{pendientes3-2}.png}\hspace*{\fill}}

Al presionar nos mostrara un modal con informacion de donde el sistema ha encontrado stock para quitar:

\noindent{\hspace*{\fill}\sphinxincludegraphics{{pendientes4}.png}\hspace*{\fill}}

Aqui podremos elejir de que farmacia quitar. A modo de ayuda tenemos colores Rojo y Verde que nos ayudan a saber si me exedi en la cantidad.
El sistema no permite que la cantidad sea mayor.

\noindent{\hspace*{\fill}\sphinxincludegraphics{{pendiente5}.png}\hspace*{\fill}}

\noindent{\hspace*{\fill}\sphinxincludegraphics{{pendiente6}.png}\hspace*{\fill}}

Una vez lista la cantidad de cada uno a quitar presionamos el boton de Registrar.

\noindent{\hspace*{\fill}\sphinxincludegraphics{{pendienteConfirm}.png}\hspace*{\fill}}

\noindent{\hspace*{\fill}\sphinxincludegraphics{{reportespedfar}.png}\hspace*{\fill}}

Esta funcionalidad cuenta con 2 modalidades:
\begin{quote}
\begin{itemize}
\item {} 
{\hyperref[pedidosfarmacia:top10\string-meds\string-pf]{\sphinxcrossref{\DUrole{std,std-ref}{Top 10 farmacias con mayor demanda de medicamentos}}}}

\item {} 
{\hyperref[pedidosfarmacia:top10\string-peds\string-pf]{\sphinxcrossref{\DUrole{std,std-ref}{Top 10 farmacias con mayor demanda de pedidos}}}}

\end{itemize}
\phantomsection\label{pedidosfarmacia:top10-meds-pf}\end{quote}


\subsubsection{Top 10 farmacias con mayor demanda de medicamentos}
\label{pedidosfarmacia:top-10-farmacias-con-mayor-demanda-de-medicamentos}\label{pedidosfarmacia:top10-meds-pf}
Si el usuario desea que los reportes se generen en base al volumen de medicamentos pedidos por cada farmacia, debera presionar la opcion \sphinxcode{Top 10 farmacias con mayor demanda de medicamentos}.

\noindent{\hspace*{\fill}\sphinxincludegraphics{{top10medspedfar}.png}\hspace*{\fill}}

Al hacerlo, se mostrara la siguiente pantalla:

\noindent{\hspace*{\fill}\sphinxincludegraphics{{pantallatop10medspedfar}.png}\hspace*{\fill}}

Si el usuario desea ajustar el rango de fecha sobre el cual se genera el reporte puede hacerlo utilizando la modalidad de filtrado por fechas:
El usuario tendrá que ingresar los parámetros de búsqueda en el formulario, y presionar el botón \sphinxcode{Filtrar}.

\begin{notice}{note}{Nota:}
Todos los campos son opcionales, de no especificarse ningún criterio de búsqueda el sistema mostrará la informacion historica completa.
\end{notice}

\noindent{\hspace*{\fill}\sphinxincludegraphics{{fechastop10medspedfar}.png}\hspace*{\fill}}

Si el usuario desea exportar el resultado generado a una planilla de Excel, debera presionar el ícono de excel.

\noindent{\hspace*{\fill}\sphinxincludegraphics{{xlstop10medspedfar}.png}\hspace*{\fill}}

Si el usuario desea exportar el resultado en un formato de imagen PNG, JPEG, PDF o SVG, debera presionar el boton de herramientas de exportacion y seleccionar la opcion correspondiente.

\noindent{\hspace*{\fill}\sphinxincludegraphics{{btnexptop10medspedfar}.png}\hspace*{\fill}}


\subsubsection{Top 10 farmacias con mayor demanda de pedidos}
\label{pedidosfarmacia:top10-peds-pf}\label{pedidosfarmacia:top-10-farmacias-con-mayor-demanda-de-pedidos}
Si el usuario desea que los reportes se generen en base a la cantidad de pedidos realizados por cada farmacia, debera presionar la opcion \sphinxcode{Top 10 farmacias con mayor demanda de pedidos}.

\noindent{\hspace*{\fill}\sphinxincludegraphics{{top10pedspedfar}.png}\hspace*{\fill}}

Al hacerlo, se mostrara la siguiente pantalla:

\noindent{\hspace*{\fill}\sphinxincludegraphics{{pantallatop10pedspedfar}.png}\hspace*{\fill}}

Si el usuario desea ajustar el rango de fecha sobre el cual se genera el reporte puede hacerlo utilizando la modalidad de filtrado por fechas:
El usuario tendrá que ingresar los parámetros de búsqueda en el formulario, y presionar el botón \sphinxcode{Filtrar}.

\begin{notice}{note}{Nota:}
Todos los campos son opcionales, de no especificarse ningún criterio de búsqueda el sistema mostrará la informacion historica completa.
\end{notice}

\noindent{\hspace*{\fill}\sphinxincludegraphics{{fechastop10pedspedfar}.png}\hspace*{\fill}}

Si el usuario desea exportar el resultado generado a una planilla de Excel, debera presionar el ícono de excel.

\noindent{\hspace*{\fill}\sphinxincludegraphics{{xlstop10pedspedfar}.png}\hspace*{\fill}}

Si el usuario desea exportar el resultado en un formato de imagen PNG, JPEG, PDF o SVG, debera presionar el boton de herramientas de exportacion y seleccionar la opcion correspondiente.

\noindent{\hspace*{\fill}\sphinxincludegraphics{{btnexptop10pedspedfar}.png}\hspace*{\fill}}


\subsection{Formulario de Búsqueda}
\label{pedidosfarmacia:formulario-de-busqueda}\label{pedidosfarmacia:formulario-busqueda-pf}
Si el usuario desea visualizar sólo aquellos \emph{Pedidos de Farmacia} que cumplan con algunos criterios en específico, deberá utilizar el formulario de búsqueda.

\noindent{\hspace*{\fill}\sphinxincludegraphics{{busquedapedfarm}.png}\hspace*{\fill}}

Este formulario cuenta con dos modalidades:
\begin{itemize}
\item {} 
Búsqueda simple: permite buscar los \emph{Pedidos de Farmacia} por farmacia.

\item {} 
Búsqueda avanzada: permite buscar los \emph{Pedidos de Farmacia} por farmacia, fecha desde, fecha hasta y estado del pedido.

\end{itemize}

\begin{notice}{note}{Nota:}
Todos los campos son opcionales, de no especificarse ningún criterio de búsqueda el sistema mostrará todos los \emph{Pedidos de Farmacia}.
\end{notice}

Podremos seleccionar desde el siguiente boton los posibles estados a buscar:

\noindent{\hspace*{\fill}\sphinxincludegraphics{{btnEstadosBusq1}.png}\hspace*{\fill}}

Al presionar veremos algo como lo siguiente:

\noindent{\hspace*{\fill}\sphinxincludegraphics{{btnEstadosBusq2}.png}\hspace*{\fill}}

El usuario tendrá que ingresar los parámetros de búsqueda en el formulario, y presionar el botón \sphinxcode{Buscar}. El sistema visualizará aquellos \emph{Pedidos de Farmacia} que cumplan con todas las condiciones especificadas.

Si el usuario desea limpiar los filtros activos, deberá presionar el boton \sphinxcode{Limpiar}.

\noindent{\hspace*{\fill}\sphinxincludegraphics{{limpiarpedfarm}.png}\hspace*{\fill}}


\section{Pedidos de Clínica}
\label{pedidosclinica:pedidos-de-clinica}\label{pedidosclinica::doc}
Se presentará una pantalla que contendrá un listado con todos los \emph{Pedidos de Clínica} que se encuentren registrados en el sistema hasta la fecha.

\noindent{\hspace*{\fill}\sphinxincludegraphics{{pedidosclinica}.png}\hspace*{\fill}}

Junto con el listado, se ofrecerán un conjunto de funcionalidades que permitirán manipular estos \emph{Pedidos de Clínica}.
Estas funcionalidades son:
\begin{itemize}
\item {} 
{\hyperref[pedidosclinica:alta\string-pc]{\sphinxcrossref{\DUrole{std,std-ref}{Alta Pedido}}}}

\item {} 
{\hyperref[pedidosclinica:ver\string-detalles\string-pc]{\sphinxcrossref{\DUrole{std,std-ref}{Ver Detalles}}}}

\item {} 
{\hyperref[pedidosclinica:ver\string-remitos\string-pc]{\sphinxcrossref{\DUrole{std,std-ref}{Ver Remitos}}}}

\item {} 
{\hyperref[pedidosclinica:formulario\string-busqueda\string-pc]{\sphinxcrossref{\DUrole{std,std-ref}{Formulario de Búsqueda}}}}

\end{itemize}


\subsection{Alta Pedido}
\label{pedidosclinica:alta-pedido}\label{pedidosclinica:alta-pc}
Si el usuario desea crear un nuevo \emph{Pedido de Clínica}, deberá presionar el botón \sphinxcode{Alta}.

\noindent{\hspace*{\fill}\sphinxincludegraphics{{btnaltapedclin}.png}\hspace*{\fill}}

A continuación el sistema lo redirigirá a la siguiente pantalla:

\noindent{\hspace*{\fill}\sphinxincludegraphics{{altapedclin}.png}\hspace*{\fill}}

En este punto el usuario deberá seleccionar la clínica que solicito el pedido, la obra social con la que trabaja, el médico auditor del pedido y la fecha en que fue solicitado a la empresa. A continuación deberá presionar el botón \sphinxcode{Crear Pedido}.

\begin{notice}{attention}{Atención:}
El sistema siempre validará que la información ingresada sea correcta. En caso de que los datos ingresados sean incorrectos el sistema lo informará.
En este punto, las posibles causas de errores son:
\begin{itemize}
\item {} 
La clínica ingresada no existe.

\item {} 
La fecha no existe.

\item {} 
La fecha ingresada esta fuera del rango válido.

\end{itemize}
\end{notice}

Una vez presionado el botón \sphinxcode{Crear Pedido}, se mostrará la siguiente pantalla:

\noindent{\hspace*{\fill}\sphinxincludegraphics{{detallespedclin}.png}\hspace*{\fill}}

Esta pantalla es la encargada de visualizar aquellos detalles que se irán asociando al \emph{Pedido de Clínica}.

\begin{notice}{important}{Importante:}
Los detalles de los \emph{Pedidos de Clínica} contendrán solo aquellos medicamentos que se encuentren en stock.
\end{notice}

Esta pantalla ofrece las siguientes funcionalidades:
\begin{itemize}
\item {} 
{\hyperref[pedidosclinica:agregar\string-detalle\string-pc]{\sphinxcrossref{\DUrole{std,std-ref}{Agregar Detalle}}}}

\item {} 
{\hyperref[pedidosclinica:modificar\string-detalle\string-pc]{\sphinxcrossref{\DUrole{std,std-ref}{Modificar Detalle}}}}

\item {} 
{\hyperref[pedidosclinica:eliminar\string-detalle\string-pc]{\sphinxcrossref{\DUrole{std,std-ref}{Eliminar Detalle}}}}

\item {} 
{\hyperref[pedidosclinica:registrar\string-pedido\string-pc]{\sphinxcrossref{\DUrole{std,std-ref}{Registrar Pedido}}}}

\end{itemize}


\subsubsection{Agregar Detalle}
\label{pedidosclinica:agregar-detalle}\label{pedidosclinica:agregar-detalle-pc}
Si el usuario desea agregar un detalle al \emph{Pedido de Clínica}, deberá presionar el botón \sphinxcode{Alta Detalle}.

\noindent{\hspace*{\fill}\sphinxincludegraphics{{btnadddetallepedclin}.png}\hspace*{\fill}}

Una vez realizado el paso anterior aparecerá la siguiente ventana emergente (modal):

\noindent{\hspace*{\fill}\sphinxincludegraphics{{newdetallepedclin}.png}\hspace*{\fill}}

En esta parte, se presentará un formulario que el usuario deberá completar para poder dar de alta un nuevo detalle.

\begin{notice}{attention}{Atención:}
El sistema siempre validará que la información ingresada sea correcta. En caso de que los datos ingresados sean incorrectos el sistema lo informará.
En este punto, las posibles causas de errores son:
\begin{itemize}
\item {} 
No se seleccionó un medicamento.

\item {} 
No se ingresó una cantidad.

\item {} 
La cantidad ingresada no posee un formato correcto.

\item {} 
La cantidad ingresada es menor a cero.

\item {} 
La cantidad ingresada supera el stock disponible para el medicamento seleccionado.

\end{itemize}
\end{notice}

Una vez completado el formulario, el usuario deberá presionar el botón \sphinxcode{Guardar} y el sistema se encargara de agregar el nuevo detalle al pedido.
El usuario podrá seguir dando de alta nuevos detalles, hasta donde considere necesario. Una vez que esto suceda deberá presionar el botón \sphinxcode{Cerrar} y la ventana emergente desaparecerá.


\subsubsection{Modificar Detalle}
\label{pedidosclinica:modificar-detalle-pc}\label{pedidosclinica:modificar-detalle}
Si el usuario desea modificar un detalle del \emph{Pedido de Clínica}, deberá seleccionar el detalle que desea actualizar y presionar el botón \sphinxcode{Modificar Detalle}.

\noindent{\hspace*{\fill}\sphinxincludegraphics{{btnupddetallepedclin}.png}\hspace*{\fill}}

Una vez realizado el paso anterior aparecerá la siguiente ventana emergente (modal):

\noindent{\hspace*{\fill}\sphinxincludegraphics{{upddetallepedclin}.png}\hspace*{\fill}}

En esta parte, se presentará un formulario con la información actual del detalle y el usuario deberá actualizar aquella que considere necesaria.

\begin{notice}{attention}{Atención:}
El sistema siempre validará que la información ingresada sea correcta. En caso de que los datos ingresados sean incorrectos el sistema lo informará.
En este punto, las posibles causas de errores son:
\begin{itemize}
\item {} 
No se ingresó una cantidad.

\item {} 
La cantidad ingresada no posee un formato correcto.

\item {} 
La cantidad ingresada es menor a cero.

\item {} 
La cantidad ingresada supera el stock disponible para el medicamento seleccionado.

\end{itemize}
\end{notice}

Una vez completado el formulario, el usuario deberá presionar el botón \sphinxcode{Guardar} y el sistema se encargará de actualizar la información de dicho detalle.


\subsubsection{Eliminar Detalle}
\label{pedidosclinica:eliminar-detalle-pc}\label{pedidosclinica:eliminar-detalle}
Si el usuario desea eliminar un detalle del \emph{Pedido de Clínica}, deberá seleccionar el detalle que desea eliminar y presionar el botón \sphinxcode{Baja Detalle}.

\noindent{\hspace*{\fill}\sphinxincludegraphics{{btndeldetallepedclin}.png}\hspace*{\fill}}

Una vez realizado el paso anterior aparecerá la siguiente ventana emergente (modal):

\noindent{\hspace*{\fill}\sphinxincludegraphics{{deldetallepedclin}.png}\hspace*{\fill}}

En esta parte el usuario deberá decidir si confirma la eliminación del detalle o no. Si desea confirmar la eliminación deberá presionar el botón \sphinxcode{Confirmar}, caso contrario, presionará el botón \sphinxcode{Cancelar}.


\subsubsection{Registrar Pedido}
\label{pedidosclinica:registrar-pedido}\label{pedidosclinica:registrar-pedido-pc}
Si el usuario desea registrar el \emph{Pedido de Clínica}, deberá presionar el botón \sphinxcode{Registrar}.

\noindent{\hspace*{\fill}\sphinxincludegraphics{{btnregpedclin}.png}\hspace*{\fill}}

\begin{notice}{attention}{Atención:}
El sistema siempre validará que la información del pedido de clínica sea correcta. En caso de que esta información sea incorrecta el sistema lo informará.
En este punto, las posibles causas de errores son:
\begin{itemize}
\item {} 
El pedido no contiene detalles

\item {} 
El pedido ya ha sido registrado anteriormente

\end{itemize}
\end{notice}

Una vez presionado el botón \sphinxcode{Registrar}, el sistema se encargará de crear el \emph{Pedido de Clínica} y se mostrará la siguiente ventana emergente (modal):

\noindent{\hspace*{\fill}\sphinxincludegraphics{{regpedclin}.png}\hspace*{\fill}}


\subsection{Ver Detalles}
\label{pedidosclinica:ver-detalles-pc}\label{pedidosclinica:ver-detalles}
Si el usuario desea ver los detalles de un \emph{Pedido de Clínica}, deberá seleccionar el botón de \textbf{Acción} asociado a dicho pedido y presionar la pestaña \sphinxcode{Ver Detalles}.

\noindent{\hspace*{\fill}\sphinxincludegraphics{{btndetallespedclin}.png}\hspace*{\fill}}

Una vez realizado el paso anterior aparecerá la siguiente ventana emergente (modal):

\noindent{\hspace*{\fill}\sphinxincludegraphics{{detallespedclin}.png}\hspace*{\fill}}

Esta ventana mostrará todos los detalles del \emph{Pedido de Clínica} seleccionado.


\subsection{Ver Remitos}
\label{pedidosclinica:ver-remitos}\label{pedidosclinica:ver-remitos-pc}
Si el usuario desea ver los remitos asociados a un \emph{Pedido de Clínica}, deberá seleccionar el botón de \textbf{Acción} asociado a dicho pedido y presionar la pestaña \sphinxcode{Ver Remitos}.

\noindent{\hspace*{\fill}\sphinxincludegraphics{{btnremitospedclin}.png}\hspace*{\fill}}

Una vez realizado el paso anterior aparecerá la siguiente ventana emergente (modal):

\noindent{\hspace*{\fill}\sphinxincludegraphics{{remitospedclin}.png}\hspace*{\fill}}

Esta ventana mostrará todos los remitos vinculados al \emph{Pedido de Clínica} seleccionado.

\begin{notice}{note}{Nota:}
En caso de que el pedido no tenga remitos asociados el sistema lo informará.
\end{notice}

El usuario tendra la opción de visualizar un remito en PDF, presionanado el boton \sphinxcode{Descargar} asociado a él.

Si se desea generar el remito en un pdf, el usuario deberá seleccionar el botón asociado al remito correspondiente y el sistema se encargará de generar el mismo.

Se vera algo como lo siguiente:

\noindent{\hspace*{\fill}\sphinxincludegraphics{{remitopedidoclinica}.png}\hspace*{\fill}}


\subsection{Reportes}
\label{pedidosclinica:reportes}\label{pedidosclinica:reportes-pc}
Si el usuario desea visualizar y/o generar reportes de estadisticas en relacion a los \emph{Pedidos de Cliínica}, debera seleccionar el boton de \textbf{Reportes}.

\noindent{\hspace*{\fill}\sphinxincludegraphics{{reportesclin}.png}\hspace*{\fill}}

Esta funcionalidad cuenta con 2 modalidades:
\begin{itemize}
\item {} 
{\hyperref[pedidosclinica:top10\string-meds\string-pc]{\sphinxcrossref{\DUrole{std,std-ref}{Top 10 clínicas con mayor demanda de medicamentos}}}}

\item {} 
{\hyperref[pedidosclinica:top10\string-peds\string-pc]{\sphinxcrossref{\DUrole{std,std-ref}{Top 10 clínicas con mayor demanda de pedidos}}}}

\end{itemize}


\subsubsection{Top 10 clínicas con mayor demanda de medicamentos}
\label{pedidosclinica:top10-meds-pc}\label{pedidosclinica:top-10-clinicas-con-mayor-demanda-de-medicamentos}
Si el usuario desea que los reportes se generen en base al volumen de medicamentos pedidos por cada clínica, debera presionar la opcion \sphinxcode{Top 10 clínicas con mayor demanda de medicamentos}.

\noindent{\hspace*{\fill}\sphinxincludegraphics{{top10medsclin}.png}\hspace*{\fill}}

Al hacerlo, se mostrara la siguiente pantalla:

\noindent{\hspace*{\fill}\sphinxincludegraphics{{pantallatop10medsclin}.png}\hspace*{\fill}}

Si el usuario desea ajustar el rango de fecha sobre el cual se genera el reporte puede hacerlo utilizando la modalidad de filtrado por fechas:
El usuario tendrá que ingresar los parámetros de búsqueda en el formulario, y presionar el botón \sphinxcode{Filtrar}.

\begin{notice}{note}{Nota:}
Todos los campos son opcionales, de no especificarse ningún criterio de búsqueda el sistema mostrará la informacion historica completa.
\end{notice}

\noindent{\hspace*{\fill}\sphinxincludegraphics{{fechastop10medsclin}.png}\hspace*{\fill}}

Si el usuario desea exportar el resultado generado a una planilla de Excel, debera presionar el ícono de excel.

\noindent{\hspace*{\fill}\sphinxincludegraphics{{xlstop10medsclin}.png}\hspace*{\fill}}

Si el usuario desea exportar el resultado en un formato de imagen PNG, JPEG, PDF o SVG, debera presionar el boton de herramientas de exportacion y seleccionar la opcion correspondiente.

\noindent{\hspace*{\fill}\sphinxincludegraphics{{btnexptop10medsclin}.png}\hspace*{\fill}}


\subsubsection{Top 10 clínicas con mayor demanda de pedidos}
\label{pedidosclinica:top-10-clinicas-con-mayor-demanda-de-pedidos}\label{pedidosclinica:top10-peds-pc}
Si el usuario desea que los reportes se generen en base a la cantidad de pedidos realizados por cada clínica, debera presionar la opcion \sphinxcode{Top 10 clínicas con mayor demanda de pedidos}.

\noindent{\hspace*{\fill}\sphinxincludegraphics{{top10pedsclin}.png}\hspace*{\fill}}

Al hacerlo, se mostrara la siguiente pantalla:

\noindent{\hspace*{\fill}\sphinxincludegraphics{{pantallatop10pedsclin}.png}\hspace*{\fill}}

Si el usuario desea ajustar el rango de fecha sobre el cual se genera el reporte puede hacerlo utilizando la modalidad de filtrado por fechas:
El usuario tendrá que ingresar los parámetros de búsqueda en el formulario, y presionar el botón \sphinxcode{Filtrar}.

\begin{notice}{note}{Nota:}
Todos los campos son opcionales, de no especificarse ningún criterio de búsqueda el sistema mostrará la informacion historica completa.
\end{notice}

\noindent{\hspace*{\fill}\sphinxincludegraphics{{fechastop10pedsclin}.png}\hspace*{\fill}}

Si el usuario desea exportar el resultado generado a una planilla de Excel, debera presionar el ícono de excel.

\noindent{\hspace*{\fill}\sphinxincludegraphics{{xlstop10pedsclin}.png}\hspace*{\fill}}

Si el usuario desea exportar el resultado en un formato de imagen PNG, JPEG, PDF o SVG, debera presionar el boton de herramientas de exportacion y seleccionar la opcion correspondiente.

\noindent{\hspace*{\fill}\sphinxincludegraphics{{btnexptop10pedsclin}.png}\hspace*{\fill}}


\subsection{Formulario de Búsqueda}
\label{pedidosclinica:formulario-busqueda-pc}\label{pedidosclinica:formulario-de-busqueda}
Si el usuario desea visualizar sólo aquellos \emph{Pedidos de Clínica} que cumplan con algunos criterios en específico, deberá utilizar el formulario de búsqueda.

\noindent{\hspace*{\fill}\sphinxincludegraphics{{busquedapedclin}.png}\hspace*{\fill}}

Este formulario cuenta con dos modalidades:
\begin{itemize}
\item {} 
Búsqueda simple: permite buscar los \emph{Pedidos de Clínica} por clínica.

\item {} 
Búsqueda avanzada: permite buscar los \emph{Pedidos de Clínica} por clínica, obra social, fecha desde y fecha hasta.

\end{itemize}

\begin{notice}{note}{Nota:}
Todos los campos son opcionales, de no especificarse ningún criterio de búsqueda el sistema mostrará todos los \emph{Pedidos de Clínica}.
\end{notice}

El usuario tendrá que ingresar los parámetros de búsqueda en el formulario, y presionar el botón \sphinxcode{Buscar}. El sistema visualizará aquellos \emph{Pedidos de Clínica} que cumplan con todas las condiciones especificadas.

Si el usuario desea limpiar los filtros activos, deberá presionar el boton \sphinxcode{Limpiar}.

\noindent{\hspace*{\fill}\sphinxincludegraphics{{limpiarpedclin}.png}\hspace*{\fill}}


\section{Pedidos a Laboratorio}
\label{pedidosalab:pedidos-a-laboratorio}\label{pedidosalab::doc}
Se presentará una pantalla que contendrá un listado con todos los \emph{Pedidos a Laboratorio} que se encuentren registrados en el sistema hasta la fecha.

\noindent{\hspace*{\fill}\sphinxincludegraphics{{btnregrecep}.png}\hspace*{\fill}}

Junto con el listado, se ofrecerán un conjunto de funcionalidades que permitirán manipular estos \emph{Pedidos a Laboratorio}.
Estas funcionalidades son:
\begin{itemize}
\item {} 
{\hyperref[pedidosalab:alta\string-pl]{\sphinxcrossref{\DUrole{std,std-ref}{Alta Pedido}}}}

\item {} 
{\hyperref[pedidosalab:cancelar\string-pedido\string-pl]{\sphinxcrossref{\DUrole{std,std-ref}{Cancelar Pedido}}}}

\item {} 
{\hyperref[pedidosalab:ver\string-detalles\string-pl]{\sphinxcrossref{\DUrole{std,std-ref}{Ver Detalles}}}}

\item {} 
{\hyperref[pedidosalab:ver\string-remitos\string-pl]{\sphinxcrossref{\DUrole{std,std-ref}{Ver Remitos}}}}

\item {} 
{\hyperref[pedidosalab:formulario\string-busqueda\string-pl]{\sphinxcrossref{\DUrole{std,std-ref}{Formulario de Búsqueda}}}}

\end{itemize}


\subsection{Alta Pedido}
\label{pedidosalab:alta-pl}\label{pedidosalab:alta-pedido}
Si el usuario desea crear un nuevo \emph{Pedido a Laboratorio}, deberá presionar el botón \sphinxcode{Alta}.

\noindent{\hspace*{\fill}\sphinxincludegraphics{{btnaltapedlab}.png}\hspace*{\fill}}

A continuación el sistema lo redirigirá a la siguiente pantalla:

\noindent{\hspace*{\fill}\sphinxincludegraphics{{altapedlab}.png}\hspace*{\fill}}

En este punto el usuario deberá seleccionar el laboratorio al cual desea realizarle el pedido. A continuación deberá presionar el botón \sphinxcode{Continuar}.

\begin{notice}{attention}{Atención:}
El sistema siempre validará que la información ingresada sea correcta. En caso de que los datos ingresados sean incorrectos el sistema lo informará.
En este punto, las posibles causas de errores son:
\begin{itemize}
\item {} 
No se seleccionó un laboratorio.

\end{itemize}
\end{notice}

Una vez presionado el botón \sphinxcode{Continuar}, se mostrará la siguiente pantalla:

\noindent{\hspace*{\fill}\sphinxincludegraphics{{detallespedlab}.png}\hspace*{\fill}}

Esta pantalla es la encargada de visualizar aquellos detalles que se irán asociando al \emph{Pedido a Laboratorio}.

\begin{notice}{note}{Nota:}
De forma automática el sistema se encargará de buscar y agregar al \emph{Pedido a Laboratorio} aquellos detalles que pertenezcan a Pedidos de Farmacia y que cumplan las siguientes condiciones:
\begin{itemize}
\item {} 
Que contengan un medicamento producido por el laboratorio al cual se le está realizando el pedido.

\item {} 
Que no haya stock suficiente para satisfacer el medicamento que compone al detalle.

\item {} 
Que no se encuentren dentro de algún otro \emph{Pedido a Laboratorio}.

\end{itemize}
\end{notice}

Esta pantalla ofrece las siguientes funcionalidades:
\begin{itemize}
\item {} 
{\hyperref[pedidosalab:agregar\string-detalle\string-pl]{\sphinxcrossref{\DUrole{std,std-ref}{Agregar Detalle}}}}

\item {} 
{\hyperref[pedidosalab:modificar\string-detalle\string-pl]{\sphinxcrossref{\DUrole{std,std-ref}{Modificar Detalle}}}}

\item {} 
{\hyperref[pedidosalab:eliminar\string-detalle\string-pl]{\sphinxcrossref{\DUrole{std,std-ref}{Eliminar Detalle}}}}

\item {} 
{\hyperref[pedidosalab:registrar\string-pedido\string-pl]{\sphinxcrossref{\DUrole{std,std-ref}{Registrar Pedido}}}}

\end{itemize}


\subsubsection{Agregar Detalle}
\label{pedidosalab:agregar-detalle-pl}\label{pedidosalab:agregar-detalle}
Si el usuario desea agregar un detalle al \emph{Pedido a Laboratorio}, deberá presionar el botón \sphinxcode{Alta Detalle}.

\noindent{\hspace*{\fill}\sphinxincludegraphics{{btnadddetallepedlab}.png}\hspace*{\fill}}

Una vez realizado el paso anterior aparecerá la siguiente ventana emergente (modal):

\noindent{\hspace*{\fill}\sphinxincludegraphics{{newdetallepedlab}.png}\hspace*{\fill}}

En esta parte, se presentará un formulario que el usuario deberá completar para poder dar de alta un nuevo detalle.

\begin{notice}{attention}{Atención:}
El sistema siempre validará que la información ingresada sea correcta. En caso de que los datos ingresados sean incorrectos el sistema lo informará.
En este punto, las posibles causas de errores son:
\begin{itemize}
\item {} 
No se seleccionó un medicamento.

\item {} 
No se ingresó una cantidad.

\item {} 
La cantidad ingresada no posee un formato correcto.

\item {} 
La cantidad ingresada es menor a cero.

\end{itemize}
\end{notice}

Una vez completado el formulario, el usuario deberá presionar el botón \sphinxcode{Guardar} y el sistema se encargara de agregar el nuevo detalle al pedido.
El usuario podrá seguir dando de alta nuevos detalles, hasta donde considere necesario. Una vez que esto suceda deberá presionar el botón \sphinxcode{Cerrar} y la ventana emergente desaparecerá.


\subsubsection{Modificar Detalle}
\label{pedidosalab:modificar-detalle}\label{pedidosalab:modificar-detalle-pl}
Si el usuario desea modificar un detalle del \emph{Pedido a Laboratorio}, deberá seleccionar el detalle que desea actualizar y presionar el botón \sphinxcode{Modificar Detalle}.

\noindent{\hspace*{\fill}\sphinxincludegraphics{{btnupddetallepedlab}.png}\hspace*{\fill}}

\begin{notice}{important}{Importante:}
\textbf{NO} se podrán actualizar aquellos detalles que se correspondan con Pedidos de Farmacia (aquellos que el sistema agrega automáticamente al ingresar a esta pantalla).
\end{notice}

Una vez realizado el paso anterior aparecerá la siguiente ventana emergente (modal):

\noindent{\hspace*{\fill}\sphinxincludegraphics{{upddetallepedlab}.png}\hspace*{\fill}}

En esta parte, se presentará un formulario con la información actual del detalle y el usuario deberá actualizar aquella que considere necesaria.

\begin{notice}{attention}{Atención:}
El sistema siempre validará que la información ingresada sea correcta. En caso de que los datos ingresados sean incorrectos el sistema lo informará.
En este punto, las posibles causas de errores son:
\begin{itemize}
\item {} 
No se ingresó una cantidad.

\item {} 
La cantidad ingresada no posee un formato correcto.

\item {} 
La cantidad ingresada es menor a cero.

\end{itemize}
\end{notice}

Una vez completado el formulario, el usuario deberá presionar el botón \sphinxcode{Guardar} y el sistema se encargará de actualizar la información de dicho detalle.


\subsubsection{Eliminar Detalle}
\label{pedidosalab:eliminar-detalle}\label{pedidosalab:eliminar-detalle-pl}
Si el usuario desea eliminar un detalle del \emph{Pedido a Laboratorio}, deberá seleccionar el detalle que desea eliminar y presionar el botón \sphinxcode{Baja Detalle}.

\noindent{\hspace*{\fill}\sphinxincludegraphics{{btndeldetallepedlab}.png}\hspace*{\fill}}

Una vez realizado el paso anterior aparecerá la siguiente ventana emergente (modal):

\noindent{\hspace*{\fill}\sphinxincludegraphics{{deldetallepedlab}.png}\hspace*{\fill}}

En esta parte el usuario deberá decidir si confirma la eliminación del detalle o no. Si desea confirmar la eliminación deberá presionar el botón \sphinxcode{Confirmar}, caso contrario, presionará el botón \sphinxcode{Cancelar}.


\subsubsection{Registrar Pedido}
\label{pedidosalab:registrar-pedido-pl}\label{pedidosalab:registrar-pedido}
Si el usuario desea registrar el \emph{Pedido a Laboratorio}, deberá presionar el botón \sphinxcode{Registrar}.

\noindent{\hspace*{\fill}\sphinxincludegraphics{{btnregpedlab}.png}\hspace*{\fill}}

\begin{notice}{attention}{Atención:}
El sistema siempre validará que la información del \emph{Pedido a Laboratorio} sea correcta. En caso de que esta información sea incorrecta el sistema lo informará.
En este punto, las posibles causas de errores son:
\begin{itemize}
\item {} 
El pedido no contiene detalles

\item {} 
El pedido ya ha sido registrado anteriormente

\end{itemize}
\end{notice}

Una vez presionado el botón \sphinxcode{Registrar}, el sistema se encargará de crear el \emph{Pedido a Laboratorio} y se mostrará la siguiente ventana emergente (modal).

\noindent{\hspace*{\fill}\sphinxincludegraphics{{regpedlab}.png}\hspace*{\fill}}


\subsection{Cancelar un Pedido}
\label{pedidosalab:cancelar-un-pedido}\label{pedidosalab:cancelar-pedido-pl}
Si el usuario desea cancelar un \emph{Pedido a Laboratorio}, deberá seleccionar el botón de \textbf{Acción} asociado a dicho pedido y presionar la pestaña \sphinxcode{Cancelar}.

\begin{notice}{important}{Importante:}
Solo se podrán cancelar aquellos \emph{Pedidos a Laboratorio} que se encuentren en un estado “Pendiente”.
\end{notice}

\noindent{\hspace*{\fill}\sphinxincludegraphics{{btncancelarpedlab}.png}\hspace*{\fill}}

Una vez realizado el paso anterior aparecerá la siguiente ventana emergente (modal):

\noindent{\hspace*{\fill}\sphinxincludegraphics{{cancelarpedlab}.png}\hspace*{\fill}}

En esta parte el usuario deberá decidir si confirma la eliminación del \emph{Pedido a Laboratorio}. Si desea confirmar la eliminación deberá presionar el botón \sphinxcode{Confirmar}, caso contrario, presionará el botón \sphinxcode{Cancelar}.


\subsection{Ver Detalles}
\label{pedidosalab:ver-detalles}\label{pedidosalab:ver-detalles-pl}
Si el usuario desea ver los detalles de un \emph{Pedido A Laboratorio}, deberá seleccionar el botón de \textbf{Acción} asociado a dicho pedido y presionar la pestaña \sphinxcode{Ver Detalles}.

\noindent{\hspace*{\fill}\sphinxincludegraphics{{btndetallespedlab}.png}\hspace*{\fill}}

Una vez realizado el paso anterior aparecerá la siguiente ventana emergente (modal):

\noindent{\hspace*{\fill}\sphinxincludegraphics{{verdetallespedlab}.png}\hspace*{\fill}}

Esta ventana mostrará todos los detalles del \emph{Pedido a Laboratorio} seleccionado.


\subsection{Ver Remitos}
\label{pedidosalab:ver-remitos}\label{pedidosalab:ver-remitos-pl}
Si el usuario desea ver los remitos asociados a un \emph{Pedido a Laboratorio}, deberá seleccionar el botón de \textbf{Acción} asociado a dicho pedido y presionar la pestaña \sphinxcode{Ver Remitos}.

\noindent{\hspace*{\fill}\sphinxincludegraphics{{btnremitospedlab}.png}\hspace*{\fill}}

Una vez realizado el paso anterior aparecerá la siguiente ventana emergente (modal):

\noindent{\hspace*{\fill}\sphinxincludegraphics{{remitospedlab}.png}\hspace*{\fill}}

Esta ventana mostrará todos los remitos vinculados al \emph{Pedido a Laboratorio} seleccionado.

\begin{notice}{note}{Nota:}
En caso de que el pedido no tenga remitos asociados el sistema lo informará.
\end{notice}

El usuario tendra la opción de visualizar un remito en PDF, presionanado el boton \sphinxcode{Descargar} asociado a él.

\noindent{\hspace*{\fill}\sphinxincludegraphics{{remitopedidolab}.png}\hspace*{\fill}}


\subsection{Reportes}
\label{pedidosalab:reportes-pl}\label{pedidosalab:reportes}
Si el usuario desea visualizar y/o generar reportes de estadisticas en relacion a los \emph{Pedidos a Laboratorio}, debera seleccionar el boton de \textbf{Reportes}.

\noindent{\hspace*{\fill}\sphinxincludegraphics{{reporteslab}.png}\hspace*{\fill}}

Esta funcionalidad cuenta con 2 modalidades:
\begin{itemize}
\item {} 
{\hyperref[pedidosalab:top10\string-meds\string-pl]{\sphinxcrossref{\DUrole{std,std-ref}{Top 10 Laboratorios con mayor solicitud de medicamentos}}}}

\item {} 
{\hyperref[pedidosalab:top10\string-peds\string-pl]{\sphinxcrossref{\DUrole{std,std-ref}{Top 10 Laboratorios con mayor solicitud de pedidos}}}}

\end{itemize}


\subsubsection{Top 10 Laboratorios con mayor solicitud de medicamentos}
\label{pedidosalab:top-10-laboratorios-con-mayor-solicitud-de-medicamentos}\label{pedidosalab:top10-meds-pl}
Si el usuario desea que los reportes se generen en base al volumen de medicamentos pedidos a cada laboratorio, debera presionar la opcion \sphinxcode{Top 10 clínicas con mayor demanda de medicamentos}.

\noindent{\hspace*{\fill}\sphinxincludegraphics{{top10medslab}.png}\hspace*{\fill}}

Al hacerlo, se mostrara la siguiente pantalla:

\noindent{\hspace*{\fill}\sphinxincludegraphics{{pantallatop10medslab}.png}\hspace*{\fill}}

Si el usuario desea ajustar el rango de fecha sobre el cual se genera el reporte puede hacerlo utilizando la modalidad de filtrado por fechas:
El usuario tendrá que ingresar los parámetros de búsqueda en el formulario, y presionar el botón \sphinxcode{Filtrar}.

\begin{notice}{note}{Nota:}
Todos los campos son opcionales, de no especificarse ningún criterio de búsqueda el sistema mostrará la informacion historica completa.
\end{notice}

\noindent{\hspace*{\fill}\sphinxincludegraphics{{fechastop10medslab}.png}\hspace*{\fill}}

Si el usuario desea exportar el resultado generado a una planilla de Excel, debera presionar el ícono de excel.

\noindent{\hspace*{\fill}\sphinxincludegraphics{{xlstop10medslab}.png}\hspace*{\fill}}

Si el usuario desea exportar el resultado en un formato de imagen PNG, JPEG, PDF o SVG, debera presionar el boton de herramientas de exportacion y seleccionar la opcion correspondiente.

\noindent{\hspace*{\fill}\sphinxincludegraphics{{btnexptop10medslab}.png}\hspace*{\fill}}


\subsubsection{Top 10 Laboratorios con mayor solicitud de pedidos}
\label{pedidosalab:top-10-laboratorios-con-mayor-solicitud-de-pedidos}\label{pedidosalab:top10-peds-pl}
Si el usuario desea que los reportes se generen en base a la cantidad de pedidos realizados por cada clínica, debera presionar la opcion \sphinxcode{Top 10 clínicas con mayor demanda de pedidos}.

\noindent{\hspace*{\fill}\sphinxincludegraphics{{top10pedslab}.png}\hspace*{\fill}}

Al hacerlo, se mostrara la siguiente pantalla:

\noindent{\hspace*{\fill}\sphinxincludegraphics{{pantallatop10pedslab}.png}\hspace*{\fill}}

Si el usuario desea ajustar el rango de fecha sobre el cual se genera el reporte puede hacerlo utilizando la modalidad de filtrado por fechas:
El usuario tendrá que ingresar los parámetros de búsqueda en el formulario, y presionar el botón \sphinxcode{Filtrar}.

\begin{notice}{note}{Nota:}
Todos los campos son opcionales, de no especificarse ningún criterio de búsqueda el sistema mostrará la informacion historica completa.
\end{notice}

\noindent{\hspace*{\fill}\sphinxincludegraphics{{fechastop10pedslab}.png}\hspace*{\fill}}

Si el usuario desea exportar el resultado generado a una planilla de Excel, debera presionar el ícono de excel.

\noindent{\hspace*{\fill}\sphinxincludegraphics{{xlstop10pedslab}.png}\hspace*{\fill}}

Si el usuario desea exportar el resultado en un formato de imagen PNG, JPEG, PDF o SVG, debera presionar el boton de herramientas de exportacion y seleccionar la opcion correspondiente.

\noindent{\hspace*{\fill}\sphinxincludegraphics{{btnexptop10pedslab}.png}\hspace*{\fill}}


\subsection{Formulario de Búsqueda}
\label{pedidosalab:formulario-de-busqueda}\label{pedidosalab:formulario-busqueda-pl}
Si el usuario desea visualizar sólo aquellos \emph{Pedidos a Laboratorio} que cumplan con algunos criterios en específico, deberá utilizar el formulario de búsqueda.

\noindent{\hspace*{\fill}\sphinxincludegraphics{{busquedapedlab}.png}\hspace*{\fill}}

Este formulario cuenta con dos modalidades:
\begin{itemize}
\item {} 
Búsqueda simple: permite buscar los \emph{Pedidos a Laboratorio} por laboratorio.

\item {} 
Búsqueda avanzada: permite buscar los \emph{Pedidos a Laboratorio} por laboratorio, fecha desde, fecha hasta.

\end{itemize}

\begin{notice}{note}{Nota:}
Todos los campos son opcionales, de no especificarse ningún criterio de búsqueda el sistema mostrará todos los \emph{Pedidos a Laboratorio}.
\end{notice}

El usuario tendrá que ingresar los parámetros de búsqueda en el formulario, y presionar el botón \sphinxcode{Buscar}. El sistema visualizará aquellos \emph{Pedidos a Laboratorio} que cumplan con todas las condiciones especificadas.

Si el usuario desea limpiar los filtros activos, deberá presionar el boton \sphinxcode{Limpiar}.

\noindent{\hspace*{\fill}\sphinxincludegraphics{{limpiarpedlab}.png}\hspace*{\fill}}


\section{Recepcion de Pedidos de Laboratorio}
\label{receppedidosdelab::doc}\label{receppedidosdelab:recepcion-de-pedidos-de-laboratorio}
Se presentará una pantalla que contendrá un listado con todos los \emph{Pedidos a Laboratorio} que no hayan sido completamente recepcionados hasta la fecha.

\noindent{\hspace*{\fill}\sphinxincludegraphics{{recepcionpedidolaboratorio}.png}\hspace*{\fill}}

Junto con el listado, se ofrecerán una un conjunto de funcionalidades que permitirán manipular estos Pedidos de Laboratorio. Estas funcionalidades son:
\begin{itemize}
\item {} 
{\hyperref[receppedidosdelab:registrar\string-recepcion\string-rpl]{\sphinxcrossref{\DUrole{std,std-ref}{Registrar Recepción}}}}

\item {} 
{\hyperref[receppedidosdelab:formulario\string-busqueda\string-rpl]{\sphinxcrossref{\DUrole{std,std-ref}{Formulario de Búsqueda}}}}

\end{itemize}


\subsection{Registrar Recepción}
\label{receppedidosdelab:registrar-recepcion-rpl}\label{receppedidosdelab:registrar-recepcion}
Si el usuario desea comenzar a registrar la recepción de un \emph{Pedido a Laboratorio}, deberá presionar el botón \sphinxcode{Registrar}.

\noindent{\hspace*{\fill}\sphinxincludegraphics{{btnregreceppedido}.png}\hspace*{\fill}}

A continuación el sistema lo redirigirá a la siguiente pantalla:

\noindent{\hspace*{\fill}\sphinxincludegraphics{{altarecep}.png}\hspace*{\fill}}

En este punto el usuario deberá ingresar los datos del remito asociado al \emph{Pedido a Laboratorio} del cual se quiere registrar recepción. Los datos solicitados son el número de remito y su correspondiente fecha. A continuación deberá presionar el botón \sphinxcode{Continuar}.

\begin{notice}{attention}{Atención:}
El sistema siempre validará que la información ingresada sea correcta. En caso de que los datos ingresados sean incorrectos el sistema lo informará.
En este punto, las posibles causas de errores son:
\begin{itemize}
\item {} 
Uno o más campos vacios.

\item {} 
El número de remito ya está cargado en el sistema.

\item {} 
La fecha no existe.

\item {} 
La fecha ingresada esta fuera del rango válido.

\end{itemize}
\end{notice}

Una vez presionado el botón \sphinxcode{Continuar}, se mostrará la siguiente pantalla:

\noindent{\hspace*{\fill}\sphinxincludegraphics{{detallesrecep}.png}\hspace*{\fill}}

Esta pantalla es la encargada de visualizar aquellos detalles asociados al \emph{Pedido a Laboratorio} que aún no han sido completamente recepcionados.

Esta pantalla ofrece las siguientes funcionalidades:
\begin{itemize}
\item {} 
{\hyperref[receppedidosdelab:recepcion\string-detalle\string-rpl]{\sphinxcrossref{\DUrole{std,std-ref}{Acusar Recepción de un Detalle}}}}

\item {} 
{\hyperref[receppedidosdelab:registrar\string-pedido\string-rpl]{\sphinxcrossref{\DUrole{std,std-ref}{Registrar Pedido}}}}

\end{itemize}


\subsubsection{Acusar Recepción de un Detalle}
\label{receppedidosdelab:acusar-recepcion-de-un-detalle}\label{receppedidosdelab:recepcion-detalle-rpl}
Si el usuario desea acusar la recepción de un detalle del \emph{Pedido a Laboratorio}, deberá presionar el botón \sphinxcode{Acusar Recepción de un Detalle} que se encuentra asociado al mismo (detalle).

\noindent{\hspace*{\fill}\sphinxincludegraphics{{btnacusardetallerecep}.png}\hspace*{\fill}}

Una vez realizado el paso anterior el sistema lo redirigirá a la siguiente pantalla:

\noindent{\hspace*{\fill}\sphinxincludegraphics{{acusardetallerecep}.png}\hspace*{\fill}}

En esta parte, se presentará el formulario que el usuario deberá completar para poder acusar la recepción del detalle seleccionado.

El sistema ofrece dos formularios para acusar recepción de un detalle:
\begin{itemize}
\item {} 
Acusar Recepción con un lote existente en el sistema.

\item {} 
Acusar Recepción con un nuevo lote.

\end{itemize}

El usuario podrá ``navegar'' entre estas dos opciones y seleccionar la que más le convenga.

\noindent{\hspace*{\fill}\sphinxincludegraphics{{acusardetallesinloterecep}.png}\hspace*{\fill}}

\begin{notice}{important}{Importante:}
El sistema habilitará estas dos opciones siempre y cuando el medicamento asociado al detalle posea lotes activos en el sistema. Es decir, si un medicamento no posee lotes registrados, el sistema solo habilitará la opción para \emph{Acusar Recepción con un nuevo lote}.
\end{notice}

Si el usuario selecciona la modalidad de \emph{Acusar Recepción con un lote existente en el sistema} deberá proceder a ingresar los datos solicitados.

\begin{notice}{attention}{Atención:}
El sistema siempre validará que la información ingresada sea correcta. En caso de que los datos ingresados sean incorrectos el sistema lo informará.
En este punto, las posibles causas de errores son:
\begin{itemize}
\item {} 
Uno o más campos vacios.

\item {} 
No se ingresó una cantidad.

\item {} 
La cantidad ingresada es superior a la cantidad pendiente del detalle.

\end{itemize}
\end{notice}

Si el usuario selecciona la modalidad de \emph{Acusar Recepción con un nuevo lote} deberá proceder a ingresar los datos solicitados (se le suman los datos correspondientes al nuevo lote).

\begin{notice}{attention}{Atención:}
El sistema siempre validará que la información ingresada sea correcta. En caso de que los datos ingresados sean incorrectos el sistema lo informará.
En este punto, las posibles causas de errores son:
\begin{itemize}
\item {} 
Uno o más campos vacios.

\item {} 
La fecha de vencimiento del lote esta en el rango que el sistema considera como ``lote vencido''.

\item {} 
La cantidad ingresada es superior a la cantidad pendiente del detalle.

\end{itemize}
\end{notice}

Una vez completado formulario elegido, el usuario tendrá dos opciones:
\begin{itemize}
\item {} 
Presionar el botón \sphinxcode{Guardar y Volver}.

\item {} 
Presionar el botón \sphinxcode{Guardar y Continuar}.

\end{itemize}

El botón \sphinxcode{Guardar y Volver} permite guardar la \emph{recepción del detalle} en el pedido y volver a la pantalla que muestra los detalles con cantidad pendiente del \emph{Pedido a Laboratorio}.

El botón \sphinxcode{Guardar y Continuar} permite guardar la \emph{recepción del detalle} en el pedido y seguir acusando recibos del mismo.

\begin{notice}{note}{Nota:}
Si el sistema detecta que el detalle ha sido completamente recepcionado automáticamente redirijirá a la pantalla anterior (la que se encarga de visualizar los detalles del \emph{Pedidos a Laboratorio} que aún no han sido completamente recepcionados). Además deshabilitará el botón \sphinxcode{Acusar Recepción de un Detalle} asociado a este detalle.
\end{notice}


\subsubsection{Registrar Pedido}
\label{receppedidosdelab:registrar-pedido-rpl}\label{receppedidosdelab:registrar-pedido}
Si el usuario desea registrar la recepción del \emph{Pedido a Laboratorio}, deberá presionar el botón \sphinxcode{Registrar}.

\noindent{\hspace*{\fill}\sphinxincludegraphics{{btnregrecep}.png}\hspace*{\fill}}

\begin{notice}{attention}{Atención:}
El sistema siempre validará que la información del \emph{Pedido a de Farmacia} sea correcta. En caso de que esta información sea incorrecta el sistema lo informará.
En este punto, las posibles causas de errores son:
\begin{itemize}
\item {} 
El pedido no contiene detalles

\item {} 
El pedido ya ha sido registrado anteriormente

\end{itemize}
\end{notice}

Una vez presionado el botón \sphinxcode{Registrar}, el sistema se encargará de actualizar los detalles del \emph{Pedido a Laboratorio} (de ser necesario también el estado del pedido) y se mostrará la siguiente ventana emergente (modal).

\noindent{\hspace*{\fill}\sphinxincludegraphics{{regrecep}.png}\hspace*{\fill}}


\subsection{Formulario de Búsqueda}
\label{receppedidosdelab:formulario-busqueda-rpl}\label{receppedidosdelab:formulario-de-busqueda}
Si el usuario desea visualizar sólo aquellos \emph{Pedidos a Laboratorio} que cumplan con algunos criterios en específico, deberá utilizar el formulario de búsqueda.

\noindent{\hspace*{\fill}\sphinxincludegraphics{{busquedarecep}.png}\hspace*{\fill}}

Este formulario cuenta con dos modalidades:
\begin{itemize}
\item {} 
Búsqueda simple: permite buscar los \emph{Pedidos a Laboratorio} (que no hayan sido completamente recepcionados) por laboratorio.

\item {} 
Búsqueda avanzada: permite buscar los \emph{Pedidos a Laboratorio} (que no hayan sido completamente recepcionados) por laboratorio, fecha desde, fecha hasta.

\end{itemize}

\begin{notice}{note}{Nota:}
Todos los campos son opcionales, de no especificarse ningún criterio de búsqueda el sistema mostrará todos los \emph{Pedidos a Laboratorio}.
\end{notice}

El usuario tendrá que ingresar los parámetros de búsqueda en el formulario, y presionar el botón \sphinxcode{Buscar}. El sistema visualizará aquellos \emph{Pedidos a Laboratorio} (que no hayan sido completamente recepcionados) que cumplan con todas las condiciones especificadas.

Si el usuario desea limpiar los filtros activos, deberá presionar el boton \sphinxcode{Limpiar}.

\noindent{\hspace*{\fill}\sphinxincludegraphics{{limpiarrecep}.png}\hspace*{\fill}}


\section{Registrar Devolución de Medicamentos Vencidos}
\label{devolucionvencidos::doc}\label{devolucionvencidos:registrar-devolucion-de-medicamentos-vencidos}
Se presentará una pantalla en la cual el usuario deberá seleccionar el laboratorio al cual desea devolverle los medicamentos vencidos. A continuación deberá presionar el botón \sphinxcode{Continuar}.

\noindent{\hspace*{\fill}\sphinxincludegraphics{{altadevolucion}.png}\hspace*{\fill}}

\begin{notice}{attention}{Atención:}
El sistema siempre validará que la información ingresada sea correcta. En caso de que los datos ingresados sean incorrectos el sistema lo informará. En este punto, las posibles causas de errores son:
\begin{itemize}
\item {} 
No se ingresó un laboratorio.

\end{itemize}
\end{notice}

\begin{notice}{note}{Nota:}
En caso de que no existan laboratorios con medicamentos vencidos, el selector no mostrara opciones.
\end{notice}

Una vez realizado el paso anterior el usuario sera redirigido a la siguiente pantalla:

\noindent{\hspace*{\fill}\sphinxincludegraphics{{detallesdevolucion}.png}\hspace*{\fill}}

Esta pantalla es la encargada de visualizar los lotes vencidos vinculados a medicamentos producidos por el laboratorio seleccionado.

Esta pantalla ofrece las siguientes funcionalidades:
\begin{itemize}
\item {} 
{\hyperref[devolucionvencidos:registrar\string-devolucion]{\sphinxcrossref{\DUrole{std,std-ref}{Registrar Devolución}}}}

\end{itemize}


\subsection{Registrar Devolución}
\label{devolucionvencidos:registrar-devolucion}\label{devolucionvencidos:id1}
Si el usuario desea registrar la devolución de medicamentos vencidos, deberá presionar el botón \sphinxcode{Registrar}.

\noindent{\hspace*{\fill}\sphinxincludegraphics{{btnregdevolucion}.png}\hspace*{\fill}}

Una vez presionado el botón \sphinxcode{Registrar}, el sistema mostrará la siguiente ventana emergente (modal).

\noindent{\hspace*{\fill}\sphinxincludegraphics{{regdevolucion}.png}\hspace*{\fill}}


\chapter{Vista Movil}
\label{vistamovil::doc}\label{vistamovil:vista-movil}
La \emph{Vista Mobil} cuenta con algunas funcionalidades para complementar el uso de la Aplicacion de escritorio y para comodidad de los encargados.
Estas funcionalidades son:
\begin{itemize}
\item {} 
{\hyperref[vistamovil:pantalla\string-princ]{\sphinxcrossref{\DUrole{std,std-ref}{Pantalla Principal}}}}

\item {} 
{\hyperref[vistamovil:medicamentos]{\sphinxcrossref{\DUrole{std,std-ref}{Medicamentos}}}}

\item {} 
{\hyperref[vistamovil:anadir\string-pedido]{\sphinxcrossref{\DUrole{std,std-ref}{Añadir a Pedido}}}}

\item {} 
{\hyperref[vistamovil:organizaciones]{\sphinxcrossref{\DUrole{std,std-ref}{Organizaciones}}}}

\item {} 
{\hyperref[vistamovil:pedidos]{\sphinxcrossref{\DUrole{std,std-ref}{Pedidos}}}}

\item {} 
{\hyperref[vistamovil:info]{\sphinxcrossref{\DUrole{std,std-ref}{Informacion Extra}}}}

\end{itemize}


\section{Pantalla Principal}
\label{vistamovil:pantalla-principal}\label{vistamovil:pantalla-princ}
Si el usuario desea utilizar la \emph{Vista Movil}, deberá hacerlo desde un celular o dispositivo movil. Y vera la siguiente pantalla:

\noindent{\hspace*{\fill}\sphinxincludegraphics{{MOBILpantallaPrincipal}.png}\hspace*{\fill}}


\section{Campos de la Pantalla Principal}
\label{vistamovil:campos-de-la-pantalla-principal}
En la pantalla principal se veran los siguientes datos:
\begin{description}
\item[{Debera elegir entre los siguientes campos de a cuerdo a lo necesitado}] \leavevmode\begin{itemize}
\item {} 
Presionar el botón \sphinxcode{Medicamentos}.

\item {} 
Presionar el botón \sphinxcode{Organizaciones}.

\item {} 
Presionar el botón \sphinxcode{Pedidos}.

\end{itemize}

\end{description}


\section{Medicamentos}
\label{vistamovil:medicamentos}\label{vistamovil:id1}
Aqui encontraremos todos los \emph{Medicamentos} cargados en el sistema.

\noindent{\hspace*{\fill}\sphinxincludegraphics{{MOBILMedicamentos}.png}\hspace*{\fill}}

Podremos realizar la \emph{Busqueda} de uno desde el filtro superior.

\noindent{\hspace*{\fill}\sphinxincludegraphics{{MOBILMedicamentosFiltro}.png}\hspace*{\fill}}

Nos mostrara una pantalla con el detalle del medicamento en particular seleccionado.

\noindent{\hspace*{\fill}\sphinxincludegraphics{{MOBIL1Medicamento}.png}\hspace*{\fill}}

\begin{notice}{attention}{Atención:}
En el caso de ser un \emph{Encargado de Pedido} o \emph{Encargado de una sucursal}, veremos un boton para añadir ese medicamento a un pedido.
\end{notice}

\noindent{\hspace*{\fill}\sphinxincludegraphics{{MOBIL1MedicamentoAñadirAPedido}.png}\hspace*{\fill}}


\section{Añadir a Pedido}
\label{vistamovil:anadir-pedido}\label{vistamovil:anadir-a-pedido}
A travez del boton de \emph{Añadir a Pedido} accederemos a una pantalla donde podremos agregar la cantidad a pedir de ese medicamento.

\noindent{\hspace*{\fill}\sphinxincludegraphics{{MOBILPedido1}.png}\hspace*{\fill}}

Una vez insertada la cantidad debemos presionar en el boton de \emph{Pedir}

\noindent{\hspace*{\fill}\sphinxincludegraphics{{MOBILPedido1Pedir}.png}\hspace*{\fill}}

Automaticamente nos mostrara un mensaje:

\noindent{\hspace*{\fill}\sphinxincludegraphics{{MOBILPedido2}.png}\hspace*{\fill}}

Ademas cuando terminemos de cargar todos los \emph{Medicamentos} que querramos debemos presionar en el boton de \emph{Finalizar Pedido}

\noindent{\hspace*{\fill}\sphinxincludegraphics{{MOBILPedido1FinalizarPedido}.png}\hspace*{\fill}}


\section{Organizaciones}
\label{vistamovil:id2}\label{vistamovil:organizaciones}
Aqui encontraremos todas las \emph{Organizaciones} cargados en el sistema.

\noindent{\hspace*{\fill}\sphinxincludegraphics{{MOBILOrganizaciones}.png}\hspace*{\fill}}

Podremos realizar la \emph{Busqueda} de uno desde el filtro superior.

\noindent{\hspace*{\fill}\sphinxincludegraphics{{MOBILOrganizacionesBusq}.png}\hspace*{\fill}}

Nos mostrara una pantalla con el detalle de la organizacion en particular seleccionada.

\noindent{\hspace*{\fill}\sphinxincludegraphics{{MOBIL1Organizacion}.png}\hspace*{\fill}}


\section{Pedidos}
\label{vistamovil:pedidos}\label{vistamovil:id3}
Aqui encontraremos todos los \emph{Pedidos} cargados en el sistema.

\noindent{\hspace*{\fill}\sphinxincludegraphics{{MOBILPedidos}.png}\hspace*{\fill}}

Podremos realizar la \emph{Busqueda} de uno desde el filtro superior.

\noindent{\hspace*{\fill}\sphinxincludegraphics{{MOBILPedidosBusq}.png}\hspace*{\fill}}

Nos mostrara una pantalla con el detalle de el pedido en particular seleccionada.

\noindent{\hspace*{\fill}\sphinxincludegraphics{{MOBILPedidos2}.png}\hspace*{\fill}}


\section{Informacion Extra}
\label{vistamovil:info}\label{vistamovil:informacion-extra}
Ademas podremos acceder a informacion acerca de la organizacion \emph{Farma}, presionando en el boton de menu a continuacion:

\noindent{\hspace*{\fill}\sphinxincludegraphics{{MOBILpantallaPrincipalMINIMENU}.png}\hspace*{\fill}}

\noindent{\hspace*{\fill}\sphinxincludegraphics{{MOBILEInfoFarma}.png}\hspace*{\fill}}

Tambien desde cualquiera de las pantallas presionando el boton de \emph{Home} en la parte superior accedemos a la pantalla principal de la aplicacion.

\noindent{\hspace*{\fill}\sphinxincludegraphics{{MOBILpantallaPrincipalMINIMENUHome}.png}\hspace*{\fill}}


\chapter{Facturacion}
\label{facturacion:facturacion}\label{facturacion::doc}
Luego de hacer “click” en  la leyenda ``Facturacion'', el sistema muestra un submenú donde el usuario puede seleccionar la organizacion sobre la cual realizar acciones.


\section{Ventas}
\label{ventas::doc}\label{ventas:ventas}
Si el usuario desea registrar la Venta de medicamentos a una Clinica, deberá presionar el sub-item Ventas.

\noindent{\hspace*{\fill}\sphinxincludegraphics{{btnVentasPrincipal}.png}\hspace*{\fill}}

En esta pantalla se vera un listado de las \emph{ventas} a \sphinxcode{clinicas que aun no esten facturadas}:

\noindent{\hspace*{\fill}\sphinxincludegraphics{{pantallaVentas1}.png}\hspace*{\fill}}

Junto con el listado, se presentarán un conjunto de funcionalidades que permitirán manipular estas \emph{Facturas de Ventas}.
\begin{description}
\item[{Estas funcionalidades son:}] \leavevmode\begin{itemize}
\item {} 
{\hyperref[ventas:generar\string-factura\string-venta]{\sphinxcrossref{\DUrole{std,std-ref}{Generar Factura de Venta}}}}

\item {} 
{\hyperref[ventas:ver\string-facturas\string-emit]{\sphinxcrossref{\DUrole{std,std-ref}{Ver Facturas Emitidas}}}}

\item {} 
{\hyperref[ventas:imprimir]{\sphinxcrossref{\DUrole{std,std-ref}{Imprimir}}}}

\item {} 
{\hyperref[ventas:ver\string-fecha\string-pago]{\sphinxcrossref{\DUrole{std,std-ref}{Ver Fecha de Pago}}}}

\item {} 
{\hyperref[ventas:registrar\string-pago]{\sphinxcrossref{\DUrole{std,std-ref}{Registrar Pago}}}}

\item {} 
{\hyperref[ventas:reportes]{\sphinxcrossref{\DUrole{std,std-ref}{Reportes}}}}

\item {} 
{\hyperref[ventas:formulario\string-busqueda\string-ventas]{\sphinxcrossref{\DUrole{std,std-ref}{Formulario de Búsqueda Ventas}}}}

\item {} 
{\hyperref[ventas:formulario\string-busqueda\string-factemit]{\sphinxcrossref{\DUrole{std,std-ref}{Formulario de Búsqueda Facturas Emitidas}}}}

\end{itemize}

\end{description}


\subsection{Generar Factura Venta}
\label{ventas:generar-factura-venta}\label{ventas:id1}
Para realizar la factura correspondiente de la \emph{Venta}, debera presionar en el boton de \sphinxcode{Accion}, la pestaña de \sphinxcode{Generar Factura}.

\noindent{\hspace*{\fill}\sphinxincludegraphics{{btnAccionGenerarFactura0}.png}\hspace*{\fill}}

Automaticamente se presentara en pantalla un cartel con el mensaje \sphinxcode{La Factura se genero correctamente}. Y de inmediato desaparecera del listado inicial.

\noindent{\hspace*{\fill}\sphinxincludegraphics{{btnAccionGenerarFactura}.png}\hspace*{\fill}}


\subsection{Ver Facturas Emitidas}
\label{ventas:ver-facturas-emit}\label{ventas:ver-facturas-emitidas}
Para ver las facturas que fueron emitidas debemos presionar en el boton de \sphinxcode{Ver Facturas Emitidas}.

\noindent{\hspace*{\fill}\sphinxincludegraphics{{btnVerFacturasEmitidas}.png}\hspace*{\fill}}

El mismo, nos conducira a otra pantalla la cual nos mostrara las \sphinxcode{*Facturas Emitidas*}.

\noindent{\hspace*{\fill}\sphinxincludegraphics{{pantallaVerFacturasEmitidas}.png}\hspace*{\fill}}

Aqui veremos tanto las facturas \sphinxcode{Pagas} como las \sphinxcode{Impagas}

\noindent{\hspace*{\fill}\sphinxincludegraphics{{facturasPagasYNoPagas}.png}\hspace*{\fill}}


\subsection{Imprimir}
\label{ventas:id2}\label{ventas:imprimir}
Para \emph{Imprimir} una factura ya Pagada, tenemos que presionar el boton de \sphinxcode{Accion} y ahi el boton de \sphinxcode{Imprimir}.

\noindent{\hspace*{\fill}\sphinxincludegraphics{{btnImprimir}.png}\hspace*{\fill}}

Nos conducira a la \emph{Factura}, lista para \sphinxcode{Imprimir}.

\noindent{\hspace*{\fill}\sphinxincludegraphics{{facturaGeneradaPago}.png}\hspace*{\fill}}


\subsection{Ver Fecha de Pago}
\label{ventas:ver-fecha-de-pago}\label{ventas:ver-fecha-pago}
Para \emph{Ver la Fecha de Pago}, debemos presionar en el boton de \sphinxcode{Accion} y ahi el boton de \sphinxcode{Ver Fecha de Pago}.

\noindent{\hspace*{\fill}\sphinxincludegraphics{{btnVerFechaPago}.png}\hspace*{\fill}}


\subsection{Registrar Pago}
\label{ventas:registrar-pago}\label{ventas:id3}
A la hora de \emph{Registrar el Pago} debemos presionar sobre la factura que aun no este pagada, el boton de \sphinxcode{Accion} y ahi el boton de \sphinxcode{Registrar Pago}.

\noindent{\hspace*{\fill}\sphinxincludegraphics{{btnRegistrarPago}.png}\hspace*{\fill}}

\noindent{\hspace*{\fill}\sphinxincludegraphics{{cartelElp}.png}\hspace*{\fill}}


\subsection{Reportes}
\label{ventas:id4}\label{ventas:reportes}
Si el usuario desea visualizar y/o generar reportes de estadisticas en relacion a las \emph{Ventas}, debera seleccionar el boton de \textbf{Reportes}.

\noindent{\hspace*{\fill}\sphinxincludegraphics{{reportespedfar}.png}\hspace*{\fill}}

Esta funcionalidad cuenta con la modalidad de:
\begin{quote}
\begin{itemize}
\item {} 
{\hyperref[ventas:top10\string-mont\string-vtas\string-pf]{\sphinxcrossref{\DUrole{std,std-ref}{Top 10 Monto Ventas}}}}

\end{itemize}
\phantomsection\label{ventas:top10-mont-vtas-pf}\end{quote}


\subsubsection{Top 10 Monto de Ventas}
\label{ventas:top-10-monto-de-ventas}\label{ventas:top10-mont-vtas-pf}
Si el usuario desea que los reportes se generen en base al volumen de ventas general, debera presionar la opcion \sphinxcode{Top 10 Monto de Ventas}.

\noindent{\hspace*{\fill}\sphinxincludegraphics{{btnEstadisticasDeVentas}.png}\hspace*{\fill}}

Al hacerlo, se mostrara la siguiente pantalla:

\noindent{\hspace*{\fill}\sphinxincludegraphics{{top10Ventas1}.png}\hspace*{\fill}}

\noindent{\hspace*{\fill}\sphinxincludegraphics{{top10Ventas2}.png}\hspace*{\fill}}

Si el usuario desea ajustar el rango de fecha sobre el cual se genera el reporte puede hacerlo utilizando la modalidad de filtrado por fechas:
El usuario tendrá que ingresar los parámetros de búsqueda en el formulario, y presionar el botón \sphinxcode{Filtrar}.

\noindent{\hspace*{\fill}\sphinxincludegraphics{{filtroTop10Ventas1}.png}\hspace*{\fill}}

Si el usuario desea exportar el resultado generado a una planilla de Excel, debera presionar el ícono de excel.

\noindent{\hspace*{\fill}\sphinxincludegraphics{{filtroTop10Ventas2}.png}\hspace*{\fill}}

Si el usuario desea exportar el resultado en un formato de imagen PNG, JPEG, PDF o SVG, debera presionar el boton de herramientas de exportacion y seleccionar la opcion correspondiente.

\noindent{\hspace*{\fill}\sphinxincludegraphics{{filtroTop10Ventas3}.png}\hspace*{\fill}}


\subsection{Formulario de Búsqueda de Ventas}
\label{ventas:formulario-busqueda-ventas}\label{ventas:formulario-de-busqueda-de-ventas}
Si el usuario desea visualizar sólo aquellas \emph{Ventas} que cumplan con algunos criterios en específico, deberá utilizar el formulario de búsqueda.

\noindent{\hspace*{\fill}\sphinxincludegraphics{{formBusqFacturasVentas}.png}\hspace*{\fill}}

Este formulario cuenta con dos modalidades:
\begin{itemize}
\item {} 
Búsqueda simple: permite buscar las \emph{Ventas} por cliente.

\item {} 
Búsqueda avanzada: permite buscar las \emph{Ventas} por razon social y fechas desde y hasta.

\end{itemize}

\begin{notice}{note}{Nota:}
Todos los campos son opcionales, de no especificarse ningún criterio de búsqueda el sistema mostrará todos las \emph{Ventas}.
\end{notice}

El usuario tendrá que ingresar los parámetros de búsqueda en el formulario, y presionar el botón \sphinxcode{Buscar}. El sistema visualizará aquellas \emph{Ventas} que cumplan con todas las condiciones especificadas.

Si el usuario desea limpiar los filtros activos, deberá presionar el boton \sphinxcode{Limpiar}.

\noindent{\hspace*{\fill}\sphinxincludegraphics{{formBusqFacturasVentasLimpiar}.png}\hspace*{\fill}}


\subsection{Formulario de Búsqueda Facturas Emitidas}
\label{ventas:formulario-de-busqueda-facturas-emitidas}\label{ventas:formulario-busqueda-factemit}
Si el usuario desea visualizar sólo aquellas \emph{Facturas Emitidas} que cumplan con algunos criterios en específico, deberá utilizar el formulario de búsqueda.

\noindent{\hspace*{\fill}\sphinxincludegraphics{{formBusqFacturasEmitidas}.png}\hspace*{\fill}}

Este formulario cuenta con dos modalidades:
\begin{itemize}
\item {} 
Búsqueda simple: permite buscar las \emph{Facturas Emitidas} por cliente.

\item {} 
Búsqueda avanzada: permite buscar las \emph{Facturas Emitidas} por numero de factura, fechas desde y hasta y tambien por facturas pagas e impagas.

\end{itemize}

\begin{notice}{note}{Nota:}
Todos los campos son opcionales, de no especificarse ningún criterio de búsqueda el sistema mostrará todos las \emph{Facturas Emitidas}.
\end{notice}

El usuario tendrá que ingresar los parámetros de búsqueda en el formulario, y presionar el botón \sphinxcode{Buscar}. El sistema visualizará aquellas \emph{Facturas Emitidas} que cumplan con todas las condiciones especificadas.

Si el usuario desea limpiar los filtros activos, deberá presionar el boton \sphinxcode{Limpiar}.

\noindent{\hspace*{\fill}\sphinxincludegraphics{{formBusqFacturasEmitidasLimpiar}.png}\hspace*{\fill}}


\section{Compras}
\label{compras::doc}\label{compras:compras}
Si el usuario desea registrar la \emph{Compra de medicamentos a un Laboratorio}, deberá presionar el sub-item \sphinxcode{Compras}.

\noindent{\hspace*{\fill}\sphinxincludegraphics{{btncompras}.png}\hspace*{\fill}}

A continuación el sistema lo redirigirá a la siguiente pantalla:

\noindent{\hspace*{\fill}\sphinxincludegraphics{{compras}.png}\hspace*{\fill}}

En esta pantalla se vera un listado de las \emph{Compras} a \sphinxcode{Laboratorios que aun no esten facturadas}:

\noindent{\hspace*{\fill}\sphinxincludegraphics{{pantallaCompras1}.png}\hspace*{\fill}}

Junto con el listado, se presentarán un conjunto de funcionalidades que permitirán manipular estas \emph{Facturas de Compras}.
\begin{description}
\item[{Estas funcionalidades son:}] \leavevmode\begin{itemize}
\item {} 
\DUrole{xref,std,std-ref}{Registrar Factura de Compra}

\item {} 
{\hyperref[compras:ver\string-facturados]{\sphinxcrossref{\DUrole{std,std-ref}{Ver Facturados}}}}

\item {} 
{\hyperref[compras:cancelar\string-pago]{\sphinxcrossref{\DUrole{std,std-ref}{Cancelar Pago}}}}

\item {} 
{\hyperref[compras:ver\string-detalle\string-pago]{\sphinxcrossref{\DUrole{std,std-ref}{Ver detalle de Pago}}}}

\item {} 
{\hyperref[ventas:reportes]{\sphinxcrossref{\DUrole{std,std-ref}{Reportes}}}}

\item {} 
{\hyperref[ventas:formulario\string-busqueda\string-ventas]{\sphinxcrossref{\DUrole{std,std-ref}{Formulario de Búsqueda Ventas}}}}

\item {} 
{\hyperref[ventas:formulario\string-busqueda\string-factemit]{\sphinxcrossref{\DUrole{std,std-ref}{Formulario de Búsqueda Facturas Emitidas}}}}

\end{itemize}

\end{description}


\subsection{Generar Factura de Compra}
\label{compras:generar-factura-compra}\label{compras:generar-factura-de-compra}
Para realizar la factura correspondiente de la \emph{Compra}, debera presionar en el boton de \sphinxcode{Accion}, la pestaña de \sphinxcode{Registrar Factura}.

\noindent{\hspace*{\fill}\sphinxincludegraphics{{btnAccionRegistrarFactura0}.png}\hspace*{\fill}}

Automaticamente se presentara en pantalla un cartel para completarlos datos correspondientes de \sphinxcode{La Factura de Compra}.

\noindent{\hspace*{\fill}\sphinxincludegraphics{{btnAccionRegistrarFactura1}.png}\hspace*{\fill}}

\noindent{\hspace*{\fill}\sphinxincludegraphics{{btnAccionRegistrarFactura2}.png}\hspace*{\fill}}

Una vez realizado esto desaparecera la factura donde estaba y se podra ver desde el boton de \sphinxcode{Facturas Generadas}.


\subsection{Ver Facturados}
\label{compras:ver-facturados}\label{compras:id1}
Para ver las facturas que fueron \emph{Facturadas} debemos presionar en el boton de \sphinxcode{Ver Facturados}.

\noindent{\hspace*{\fill}\sphinxincludegraphics{{btnVerFacturados}.png}\hspace*{\fill}}

El mismo, nos conducira a otra pantalla la cual nos mostrara las \sphinxcode{*Facturas Emitidas*}.

En el caso de que no contenga facturas mostrara un mensaje que diga que ``LA TABLA ESTA VACIA''

\noindent{\hspace*{\fill}\sphinxincludegraphics{{pantallaVerFacturasEmitidasFacturadas}.png}\hspace*{\fill}}

Sino, mostrara las facturas ya emitidas.

\noindent{\hspace*{\fill}\sphinxincludegraphics{{facturasEmitidasCompras}.png}\hspace*{\fill}}

En el boton de \sphinxcode{Accion} encontraremos un submenu con un boton de \sphinxcode{Ver Factura}.

\noindent{\hspace*{\fill}\sphinxincludegraphics{{btnVerFacturasEmitidas2}.png}\hspace*{\fill}}

Una vez presionado el boton nos aparecera una pantalla con los datos de la factura en cuestion.

\noindent{\hspace*{\fill}\sphinxincludegraphics{{facturaDeCompraImpaga}.png}\hspace*{\fill}}

Si queremos realizar el pago debemos presionar el boton de \sphinxcode{Realizar Pago}.

\noindent{\hspace*{\fill}\sphinxincludegraphics{{btnRealizarPagoCompra}.png}\hspace*{\fill}}

Nos aparecera una pantalla para que carguemos la informacion respectiva de la factura:
Tener en cuenta que el sistema corrobora la informacion ingresada

\noindent{\hspace*{\fill}\sphinxincludegraphics{{infoErrorPagoFacturaCompra}.png}\hspace*{\fill}}

Una vez colocada la informacion respectiva precionamos el boton de \sphinxcode{Pagar}
Y seguido de esto nos aparecera un mensaje de exito.

\noindent{\hspace*{\fill}\sphinxincludegraphics{{pagoCorrectoCompra}.png}\hspace*{\fill}}


\subsection{Cancelar Pago}
\label{compras:cancelar-pago}\label{compras:id2}
Para \sphinxcode{Cancelar Pago} una vez pagado, debemos presionar el boton de \emph{Cancelar Pago}.

\noindent{\hspace*{\fill}\sphinxincludegraphics{{VerPagoOCancelarPago2}.png}\hspace*{\fill}}

Al presionarlo el sistema mostrara un mensaje de confirmacion.

\noindent{\hspace*{\fill}\sphinxincludegraphics{{cancelarPagoCompra1}.png}\hspace*{\fill}}

\noindent{\hspace*{\fill}\sphinxincludegraphics{{cancelarPagoCompra2}.png}\hspace*{\fill}}


\subsection{Ver Detalle de Pago}
\label{compras:ver-detalle-de-pago}\label{compras:ver-detalle-pago}
Para Ver el detalle de un pago realizado, debemos presionar el boton de \emph{Ver Pago}.

\noindent{\hspace*{\fill}\sphinxincludegraphics{{VerPagoOCancelarPago1}.png}\hspace*{\fill}}

\noindent{\hspace*{\fill}\sphinxincludegraphics{{verDetalleDePagoCompra}.png}\hspace*{\fill}}


\subsection{Reportes}
\label{compras:id3}\label{compras:reportes}
Si el usuario desea visualizar y/o generar reportes de estadisticas en relacion a las \emph{Ventas}, debera seleccionar el boton de \textbf{Reportes}.

\noindent{\hspace*{\fill}\sphinxincludegraphics{{reportespedfar}.png}\hspace*{\fill}}

Esta funcionalidad cuenta con la modalidad de:
\begin{quote}
\begin{itemize}
\item {} 
{\hyperref[compras:top10\string-mont\string-compr\string-pf]{\sphinxcrossref{\DUrole{std,std-ref}{Top 10 Monto de Compras a Laboratorio}}}}

\end{itemize}
\phantomsection\label{compras:top10-mont-compr-pf}\end{quote}


\subsubsection{Top 10 Monto de Compras a Laboratorio}
\label{compras:top-10-monto-de-compras-a-laboratorio}\label{compras:top10-mont-compr-pf}
Si el usuario desea que los reportes se generen en base al volumen de ventas general, debera presionar la opcion \sphinxcode{Top 10 Monto de Compras a Laboratorio}.

\noindent{\hspace*{\fill}\sphinxincludegraphics{{top10medspedfar}.png}\hspace*{\fill}}

Al hacerlo, se mostrara la siguiente pantalla:

\noindent{\hspace*{\fill}\sphinxincludegraphics{{pantallatop10medspedfar}.png}\hspace*{\fill}}

Si el usuario desea ajustar el rango de fecha sobre el cual se genera el reporte puede hacerlo utilizando la modalidad de filtrado por fechas:
El usuario tendrá que ingresar los parámetros de búsqueda en el formulario, y presionar el botón \sphinxcode{Filtrar}.

\begin{notice}{note}{Nota:}
Todos los campos son opcionales, de no especificarse ningún criterio de búsqueda el sistema mostrará la informacion historica completa.
\end{notice}

\noindent{\hspace*{\fill}\sphinxincludegraphics{{fechastop10medspedfar}.png}\hspace*{\fill}}

Si el usuario desea exportar el resultado generado a una planilla de Excel, debera presionar el ícono de excel.

\noindent{\hspace*{\fill}\sphinxincludegraphics{{xlstop10medspedfar}.png}\hspace*{\fill}}

Si el usuario desea exportar el resultado en un formato de imagen PNG, JPEG, PDF o SVG, debera presionar el boton de herramientas de exportacion y seleccionar la opcion correspondiente.

\noindent{\hspace*{\fill}\sphinxincludegraphics{{btnexptop10medspedfar}.png}\hspace*{\fill}}


\subsection{Formulario de Búsqueda de Compras}
\label{compras:formulario-busqueda-ventas}\label{compras:formulario-de-busqueda-de-compras}
Si el usuario desea visualizar sólo aquellas \emph{Compras} que cumplan con algunos criterios en específico, deberá utilizar el formulario de búsqueda.

\noindent{\hspace*{\fill}\sphinxincludegraphics{{formBusqFacturasCompras}.png}\hspace*{\fill}}

Este formulario cuenta con dos modalidades:
\begin{itemize}
\item {} 
Búsqueda simple: permite buscar las \emph{Compras} por proveedor.

\item {} 
Búsqueda avanzada: permite buscar las \emph{Compras} por fechas desde y hasta.

\end{itemize}

\begin{notice}{note}{Nota:}
Todos los campos son opcionales, de no especificarse ningún criterio de búsqueda el sistema mostrará todos las \emph{Compras}.
\end{notice}

El usuario tendrá que ingresar los parámetros de búsqueda en el formulario, y presionar el botón \sphinxcode{Buscar}. El sistema visualizará aquellas \emph{Compras} que cumplan con todas las condiciones especificadas.

Si el usuario desea limpiar los filtros activos, deberá presionar el boton \sphinxcode{Limpiar}.

\noindent{\hspace*{\fill}\sphinxincludegraphics{{formBusqFacturasCompras2}.png}\hspace*{\fill}}


\subsection{Formulario de Búsqueda Facturas Emitidas}
\label{compras:formulario-de-busqueda-facturas-emitidas}\label{compras:formulario-busqueda-factemit}
Si el usuario desea visualizar sólo aquellas \emph{Facturas Emitidas} que cumplan con algunos criterios en específico, deberá utilizar el formulario de búsqueda.

\noindent{\hspace*{\fill}\sphinxincludegraphics{{pantallaVerFacturasEmitidasFacturadasBusq}.png}\hspace*{\fill}}

Este formulario cuenta con dos modalidades:
\begin{itemize}
\item {} 
Búsqueda simple: permite buscar las \emph{Facturas Emitidas} por proveedor.

\item {} 
Búsqueda avanzada: permite buscar las \emph{Facturas Emitidas} por numero de factura, fechas desde y hasta.

\end{itemize}

\begin{notice}{note}{Nota:}
Todos los campos son opcionales, de no especificarse ningún criterio de búsqueda el sistema mostrará todos las \emph{Facturas Emitidas}.
\end{notice}

El usuario tendrá que ingresar los parámetros de búsqueda en el formulario, y presionar el botón \sphinxcode{Buscar}. El sistema visualizará aquellas \emph{Facturas Emitidas} que cumplan con todas las condiciones especificadas.

Si el usuario desea limpiar los filtros activos, deberá presionar el boton \sphinxcode{Limpiar}.

\noindent{\hspace*{\fill}\sphinxincludegraphics{{pantallaVerFacturasEmitidasFacturadasBusq2}.png}\hspace*{\fill}}


\section{Formas de Pago}
\label{formasdepago::doc}\label{formasdepago:formas-de-pago}
Si el usuario desea crear una nueva \emph{Forma de Pago}, deberá presionar el sub-item \sphinxcode{Forma de Pago}.

\noindent{\hspace*{\fill}\sphinxincludegraphics{{selecformadepago}.png}\hspace*{\fill}}

A continuación el sistema lo redirigirá a la siguiente pantalla:

\noindent{\hspace*{\fill}\sphinxincludegraphics{{formadepago}.png}\hspace*{\fill}}

En esta parte el usuario se le presentará un formulario y deberá ingresar los datos solicitados para dar de alta una nueva \emph{Forma de Pago}.

\begin{notice}{attention}{Atención:}
El sistema siempre validará que la información ingresada sea correcta. En caso de que los datos ingresados sean incorrectos el sistema lo informará.
En este punto, las posibles causas de errores son:
\begin{itemize}
\item {} 
La Forma de Pago ya existe.

\end{itemize}
\end{notice}

Una vez completado el formulario, el usuario tendrá que presionar el boton \sphinxcode{Guardar} y el sistema se encargará de dar de alta la nueva Forma de Pago.


\chapter{Stock Distribuido}
\label{stockdistribuido::doc}\label{stockdistribuido:stock-distribuido}
Se presentará una pantalla que contendrá, en el caso de que hubiera, un listado con todos los \emph{Medicamentos en Stock Distribuido} que se encuentren registrados en el sistema y esten en alguna sucursalhasta la fecha.

\noindent{\hspace*{\fill}\sphinxincludegraphics{{pantallaStockDistribuido}.png}\hspace*{\fill}}

Junto con el listado, se ofrecerán un conjunto de funcionalidades que permitirán realizar busquedas sobre estos \emph{Medicamentos en Stock Distribuido}
\begin{description}
\item[{Estas funcionalidades son:}] \leavevmode\begin{itemize}
\item {} 
{\hyperref[stockdistribuido:formulario\string-busqueda\string-stock]{\sphinxcrossref{\DUrole{std,std-ref}{Formulario de Búsqueda}}}}

\end{itemize}

\end{description}


\section{Formulario de Búsqueda}
\label{stockdistribuido:formulario-de-busqueda}\label{stockdistribuido:formulario-busqueda-stock}
Si el usuario desea visualizar sólo aquellos \emph{Medicamentos en Stock Distribuido} que cumplan con algunos criterios en específico, deberá utilizar el formulario de búsqueda.

\noindent{\hspace*{\fill}\sphinxincludegraphics{{busqStockDistribuido}.png}\hspace*{\fill}}

Este formulario cuenta con la opción de búsqueda simple en base a: \sphinxcode{Farmacia}, \sphinxcode{Numero de Lote} o \sphinxcode{Medicamento}

\begin{notice}{note}{Nota:}
Este campo es opcional, de no especificarse ningún criterio de búsqueda el sistema mostrará todos las \emph{Medicamentos en Stock Distribuido}.
\end{notice}

El usuario tendrá que ingresar los parámetros de búsqueda en el formulario, y presionar el botón \sphinxcode{Buscar}. El sistema visualizará aquellos \emph{Medicamentos en Stock Distribuido} que cumplan con todas las condiciones especificadas.


\chapter{Informe de Ventas}
\label{informedeventas::doc}\label{informedeventas:informe-de-ventas}
Se presentará una pantalla que contendrá un lugar donde se podra subir un archivo que debera contener el \emph{Informe de Ventas} realizadas ese dia.

\noindent{\hspace*{\fill}\sphinxincludegraphics{{informedeventas1}.png}\hspace*{\fill}}

Aqui se debera presionar el boton de \emph{Examinar}.

\noindent{\hspace*{\fill}\sphinxincludegraphics{{btninformedeventasexaminar}.png}\hspace*{\fill}}

Se abrira una ventana de navegacion de escritorio, con la cual se podra recorrer las carpetas locales para buscar el archivo a cargar al Sistema.

\noindent{\hspace*{\fill}\sphinxincludegraphics{{informedeventas2}.png}\hspace*{\fill}}

Una vez encontrado el archivo se selecciona y abre.

\noindent{\hspace*{\fill}\sphinxincludegraphics{{informedeventas3}.png}\hspace*{\fill}}

Como ultimo paso se tendra que presionar el boton de \emph{Registrar}.

\noindent{\hspace*{\fill}\sphinxincludegraphics{{btnregistrarinfoventas}.png}\hspace*{\fill}}

\begin{notice}{note}{Nota:}
Si se produjo un error se mostrara un mensaje, causado por:
\begin{itemize}
\item {} 
Archivo no compatible.

\item {} 
El archivo no puede leerse.

\item {} 
El archivo contiene informacion erronea.

\end{itemize}

\noindent{\hspace*{\fill}\sphinxincludegraphics{{error1infoventas}.png}\hspace*{\fill}}
\end{notice}



\renewcommand{\indexname}{Índice}
\printindex
\end{document}
